\documentclass[11pt]{article}

\usepackage{hyperref}
\usepackage{amssymb}
\usepackage{bbold}

\pdfpagewidth 8.5in
\pdfpageheight 11in

\oddsidemargin -0.25in
\textwidth 7.0in
\topmargin -0.75in
\textheight 9.5in



\title{\bf GMCALC: a calculator for the Georgi-Machacek model\footnote{Code available from http://people.physics.carleton.ca/$\sim$logan/gmcalc/ .}}

\author{Current authors: Ameen Ismail$^{1,2}$, Ben Keeshan$^1$, Heather E.\ Logan$^1$\thanks{\tt logan@physics.carleton.ca}, and Yongcheng Wu$^1$ \\
Contributors: C\'eline Degrande$^3$, Katy Hartling$^1$,  Kunal Kumar$^1$, \\
Andrea D.\ Peterson$^1$, and Mark B.\ Reimer$^1$ \\
{\it $^1$Ottawa-Carleton Institute for Physics, Carleton University,} \\ 
{\it 1125 Colonel By Drive, Ottawa K1S 5B6 Canada}\\
{\it $^2$ Department of Physics, Cornell University, Ithaca, New York 14850, USA} \\
{\it $^3$CERN, Theory Division, Geneva 23 CH-1211, Switzerland} }

\date{Version 1.5.3: August 4, 2022}

\begin{document}
\maketitle

\begin{abstract}
The Georgi-Machacek model adds scalar triplets to the Standard Model Higgs sector in such a way as to preserve custodial SU(2) symmetry in the scalar potential. This allows the triplets to have a non-negligible vacuum expectation value while satisfying constraints from the $\rho$ parameter. Depending on the parameters, the 125~GeV neutral Higgs particle can have couplings to $WW$ and $ZZ$ larger than in the Standard Model due to mixing with the triplets. The model also contains singly- and doubly-charged Higgs particles that couple to vector boson pairs at tree level ($WZ$ and like-sign $WW$, respectively).

GMCALC is a FORTRAN program that, given a set of input parameters, calculates the particle spectrum and tree-level couplings in the Georgi-Machacek model, computes the couplings and decay branching ratios of the scalars, and checks theoretical, indirect, and direct experimental constraints. It also generates param\_card.dat files for MadGraph5 or MadGraph5\_aMC@NLO to be used with the corresponding FeynRules model implementation. 
\end{abstract}

\newpage
\tableofcontents
\newpage

%%%%%%%%%%%%%%%%%%%%%%%%%%%%%%%%%%%%%%%%%%%%%%%%
\section{Introduction}

The Georgi-Machacek (GM) model~\cite{Georgi:1985nv,Chanowitz:1985ug} is an extension of the Standard Model (SM) Higgs sector containing additional scalars in the triplet representation of SU(2)$_L$.  The particle content is such that an additional global SU(2)$_R$ symmetry can be imposed by hand on the scalar potential.  This ensures that the custodial SU(2) symmetry, which fixes $\rho \equiv M_W^2/M_Z^2 \cos^2\theta_W = 1$ at tree level in the SM, is preserved after electroweak symmetry breaking.  

Without the stringent constraint from the $\rho$ parameter, the vacuum expectation value (vev) of the triplets can be large, leading to interesting phenomenology.  In particular, depending on the parameters, the 125~GeV neutral Higgs particle can have couplings to $WW$ and $ZZ$ larger than in the SM due to mixing with the triplets. The model also contains singly- and doubly-charged Higgs particles that couple to vector boson pairs at tree level, leading to $H_5^+ \to W^+Z$ and like-sign $H_5^{++} \to W^+ W^+$ signatures.  Such an $H^+ W^- Z$ coupling is absent at tree level in two Higgs doublet models (2HDMs), and the $H^{++} W^- W^-$ coupling is severely suppressed in triplet models without custodial symmetry in which the triplet vev is forced to be very small by the experimental constraint from the $\rho$ parameter.

This manual describes the FORTRAN code GMCALC.  Given a set of model parameters, GMCALC calculates the mass spectrum and relevant mixing angles in the scalar sector, as well as the tree-level couplings of the scalars.  It also checks that theoretical constraints from perturbative unitarity of the quartic scalar couplings, bounded-from-belowness of the scalar potential, and the absence of deeper custodial-symmetry-breaking minima are satisfied.  The code also checks consistency of the parameter point with indirect experimental constraints from the $S$ parameter, $b \to s \gamma$, and $B_s^0 \to \mu^+\mu^-$.  It also checks consistency with direct experimental searches for additional Higgs bosons through an interface to HiggsBounds~5.2.0~\cite{Bechtle:2013wla} together with a few additional processes implemented in GMCALC itself, and with the LHC measurements of the signal strengths of the 125~GeV Higgs boson through an interface to HiggsSignals~2.2.1~\cite{Bechtle:2013xfa}.  Finally, it computes the couplings, branching ratios and total widths of the scalars.  Most of the code is based on our work in Refs.~\cite{HKL,indirect,loops}.

GMCALC includes a routine to generate param\_card.dat files for MadGraph5 to be used with the corresponding FeynRules~\cite{Alloul:2013bka} model implementation.  The FeynRules implementation for the Georgi-Machacek model, as well as a Universal FeynRules Output (UFO)~\cite{Degrande:2011ua} file for use with the MadGraph5\_aMC@NLO framework~\cite{Alwall:2014hca} including automatic calculation of the next-to-leading order QCD corrections, can be downloaded from the model database at 
\url{http://feynrules.irmp.ucl.ac.be/wiki/GeorgiMachacekModel}.

This manual is organized as follows.  In Sec.~\ref{sec:model} we give a brief description of the GM model and set our notation.  In Sec.~\ref{sec:thy} we review the theoretical constraints and their implementation.  In Sec.~\ref{sec:indir} we describe the indirect experimental constraints that are implemented in the code.  In Sec.~\ref{sec:decays} we summarize the computation of the decay partial widths of the scalars and specify the approximations made in the code.  In Sec.~\ref{sec:direct} we describe the direct experimental constraints.  Finally in Sec.~\ref{sec:using} we give instructions for using the GMCALC code.


%%%%%%%%%%%%%%%%%%%%%%%%%%%%%%%%%%%%%%%%%%%%%%%%
\section{Georgi-Machacek model}
\label{sec:model}

\subsection{Scalar potential}

The scalar sector of the Georgi-Machacek model consists of the usual complex doublet $(\phi^+,\phi^0)$ with hypercharge\footnote{We use $Q = T^3 + Y/2$.} $Y = 1$, a real 
triplet $(\xi^+,\xi^0,\xi^-)$ with $Y = 0$, and  a complex triplet $(\chi^{++},\chi^+,\chi^0)$ with $Y=2$.  The doublet is responsible for the fermion masses as in the SM.
In order to make the global SU(2)$_L \times$SU(2)$_R$ symmetry explicit, we write the doublet in the form of a bi-doublet $\Phi$ and combine the triplets to form a bi-triplet $X$:
\begin{eqnarray}
	\Phi &=& \left( \begin{array}{cc}
	\phi^{0*} &\phi^+  \\
	-\phi^{+*} & \phi^0  \end{array} \right), \\
	X &=&
	\left(
	\begin{array}{ccc}
	\chi^{0*} & \xi^+ & \chi^{++} \\
	 -\chi^{+*} & \xi^{0} & \chi^+ \\
	 \chi^{++*} & -\xi^{+*} & \chi^0  
	\end{array}
	\right).
	\label{eq:PX}
\end{eqnarray}
The vevs are defined by $\langle \Phi  \rangle = \frac{ v_{\phi}}{\sqrt{2}} \mathbb{1}_{2\times2}$  and $\langle X \rangle = v_{\chi} \mathbb{1}_{3 \times 3}$, where the Fermi constant constrains
\begin{equation}
	v_{\phi}^2 + 8 v_{\chi}^2 \equiv v^2 = \frac{1}{\sqrt{2} G_F} \approx (246~{\rm GeV})^2.
	\label{eq:vevrelation}
\end{equation} 
Note that the two triplet fields $\chi^0$ and $\xi^0$ must obtain the same vev in order to preserve custodial SU(2).
Furthermore we will decompose the neutral fields into real and imaginary parts according to
\begin{equation}
	\phi^0 \to \frac{v_{\phi}}{\sqrt{2}} + \frac{\phi^{0,r} + i \phi^{0,i}}{\sqrt{2}},
	\qquad
	\chi^0 \to v_{\chi} + \frac{\chi^{0,r} + i \chi^{0,i}}{\sqrt{2}}, 
	\qquad
	\xi^0 \to v_{\chi} + \xi^0,
\end{equation}
where we note that $\xi^0$ is already a real field.

Using the notation of Ref.~\cite{HKL}, the most general gauge-invariant scalar potential involving these fields that conserves custodial SU(2) is given by
\begin{eqnarray}
	V(\Phi,X) &= & \frac{\mu_2^2}{2} {\rm Tr}(\Phi^\dagger \Phi) 
	+  \frac{\mu_3^2}{2}  {\rm Tr}(X^\dagger X)  
	+ \lambda_1 [{\rm Tr}(\Phi^\dagger \Phi)]^2  
	+ \lambda_2 {\rm Tr}(\Phi^\dagger \Phi) {\rm Tr}(X^\dagger X)   \nonumber \\
          & & + \lambda_3 {\rm Tr}(X^\dagger X X^\dagger X)  
          + \lambda_4 [{\rm Tr}(X^\dagger X)]^2 
           - \lambda_5 {\rm Tr}( \Phi^\dagger \tau^a \Phi \tau^b) {\rm Tr}( X^\dagger t^a X t^b) 
           \nonumber \\
           & & - M_1 {\rm Tr}(\Phi^\dagger \tau^a \Phi \tau^b)(U X U^\dagger)_{ab}  
           -  M_2 {\rm Tr}(X^\dagger t^a X t^b)(U X U^\dagger)_{ab}.
           \label{eq:potential}
\end{eqnarray} 
(A translation table to other notations used in the literature is given in the appendix of Ref.~\cite{HKL}.)
Here the SU(2) generators for the doublet representation are $\tau^a = \sigma^a/2$ with $\sigma^a$ being the Pauli matrices,
the generators for the triplet representation are
\begin{equation}
	t^1= \frac{1}{\sqrt{2}} \left( \begin{array}{ccc}
	 0 & 1  & 0  \\
	  1 & 0  & 1  \\
	  0 & 1  & 0 \end{array} \right), \quad  
	  t^2= \frac{1}{\sqrt{2}} \left( \begin{array}{ccc}
	 0 & -i  & 0  \\
	  i & 0  & -i  \\
	  0 & i  & 0 \end{array} \right), \quad 
	t^3= \left( \begin{array}{ccc}
	 1 & 0  & 0  \\
	  0 & 0  & 0  \\
	  0 & 0 & -1 \end{array} \right),
\end{equation}
and the matrix $U$, which rotates $X$ into the Cartesian basis, is given by
\begin{equation}
	 U = \left( \begin{array}{ccc}
	- \frac{1}{\sqrt{2}} & 0 &  \frac{1}{\sqrt{2}} \\
	 - \frac{i}{\sqrt{2}} & 0  &   - \frac{i}{\sqrt{2}} \\
	   0 & 1 & 0 \end{array} \right).
	 \label{eq:U}
\end{equation}
We note that all the operators in Eq.~(\ref{eq:potential}) are manifestly Hermitian, so that the parameters in the scalar potential must all be real.  Explicit CP violation is thus not possible in the Georgi-Machacek model.  

\subsection{Electroweak symmetry breaking and physical spectrum}

Minimizing the scalar potential yields the following constraints:
\begin{eqnarray}
	0 = \frac{\partial V}{\partial v_{\phi}} &=& 
	v_{\phi} \left[ \mu_2^2 + 4 \lambda_1 v_{\phi}^2 
	+ 3 \left( 2 \lambda_2 - \lambda_5 \right) v_{\chi}^2 - \frac{3}{2} M_1 v_{\chi} \right], 
		\label{eq:phimincond} \\
	0 = \frac{\partial V}{\partial v_{\chi}} &=& 
	3 \mu_3^2 v_{\chi} + 3 \left( 2 \lambda_2 - \lambda_5 \right) v_{\phi}^2 v_{\chi}
	+ 12 \left( \lambda_3 + 3 \lambda_4 \right) v_{\chi}^3
	- \frac{3}{4} M_1 v_{\phi}^2 - 18 M_2 v_{\chi}^2.
	\label{eq:chimincond}
\end{eqnarray}
Inserting $v_{\phi}^2 = v^2 - 8 v_{\chi}^2$ [Eq.~(\ref{eq:vevrelation})] into Eq.~(\ref{eq:chimincond}) yields a cubic equation for $v_{\chi}$ in terms of $v$, $\mu_3^2$, $\lambda_2$, $\lambda_3$, $\lambda_4$, $\lambda_5$, $M_1$, and $M_2$.  With $v_{\chi}$ (and hence $v_{\phi}$) in hand, Eq.~(\ref{eq:phimincond}) can be used to eliminate $\mu_2^2$ in terms of the parameters in the previous sentence together with $\lambda_1$.  We illustrate below how $\lambda_1$ can also be eliminated in favor of one of the custodial singlet Higgs masses $m_h$ or $m_H$ [see Eq.~(\ref{eq:lambda1})].

The physical field content is as follows.  The Goldstone bosons are given by
\begin{eqnarray}
	G^+ &=& c_H \phi^+ + s_H \frac{\left(\chi^++\xi^+\right)}{\sqrt{2}}, \nonumber\\
	G^0  &=& c_H \phi^{0,i} + s_H \chi^{0,i},
\end{eqnarray}
where
\begin{equation}
	c_H \equiv \cos\theta_H = \frac{v_{\phi}}{v}, \qquad
	s_H \equiv \sin\theta_H = \frac{2\sqrt{2}\,v_\chi}{v}.
\end{equation}
The physical fields can be organized by their transformation properties under the custodial SU(2) symmetry into a fiveplet, a triplet, and two singlets.  The fiveplet and triplet states are given by
\begin{eqnarray}
	H_5^{++}  &=&  \chi^{++}, \nonumber\\
	H_5^+ &=& \frac{\left(\chi^+ - \xi^+\right)}{\sqrt{2}}, \nonumber\\
	H_5^0 &=& \sqrt{\frac{2}{3}} \xi^0 - \sqrt{\frac{1}{3}} \chi^{0,r}, \nonumber\\
	H_3^+ &=& - s_H \phi^+ + c_H \frac{\left(\chi^++\xi^+\right)}{\sqrt{2}}, \nonumber\\
	H_3^0 &=& - s_H \phi^{0,i} + c_H \chi^{0,i}.
\end{eqnarray}
Within each custodial multiplet, the masses are degenerate at tree level.  Using Eqs.~(\ref{eq:phimincond}--\ref{eq:chimincond}) to eliminate $\mu_2^2$ and $\mu_3^2$, the fiveplet and triplet masses can be written as
\begin{eqnarray}
	m_5^2 &=& \frac{M_1}{4 v_{\chi}} v_\phi^2 + 12 M_2 v_{\chi} 
	+ \frac{3}{2} \lambda_5 v_{\phi}^2 + 8 \lambda_3 v_{\chi}^2, \nonumber \\
	m_3^2 &=&  \frac{M_1}{4 v_{\chi}} (v_\phi^2 + 8 v_{\chi}^2) 
	+ \frac{\lambda_5}{2} (v_{\phi}^2 + 8 v_{\chi}^2) 
	= \left(  \frac{M_1}{4 v_{\chi}} + \frac{\lambda_5}{2} \right) v^2.
\end{eqnarray}
Note that the ratio $M_1/v_{\chi}$ is finite in the limit $v_{\chi} \to 0$, as can be seen from Eq.~(\ref{eq:chimincond}) which yields
\begin{equation}
	\frac{M_1}{v_{\chi}} = \frac{4}{v_{\phi}^2} 
	\left[ \mu_3^2 + (2 \lambda_2 - \lambda_5) v_{\phi}^2 
	+ 4(\lambda_3 + 3 \lambda_4) v_{\chi}^2 - 6 M_2 v_{\chi} \right].
\end{equation}

The two custodial SU(2) singlets are given in the gauge basis by
\begin{eqnarray}
	H_1^0 &=& \phi^{0,r}, \nonumber \\
	H_1^{0 \prime} &=& \sqrt{\frac{1}{3}} \xi^0 + \sqrt{\frac{2}{3}} \chi^{0,r}.
\end{eqnarray}
These states mix by an angle $\alpha$ to form the two custodial-singlet mass eigenstates $h$ and $H$, defined such that $m_h < m_H$:
\begin{eqnarray}
	h &=& \cos \alpha \, H_1^0 - \sin \alpha \, H_1^{0\prime},   \\ \nonumber 
	H &=& \sin \alpha \, H_1^0 + \cos \alpha \, H_1^{0\prime},
\end{eqnarray}
and we will abbreviate $c_{\alpha} = \cos\alpha$, $s_{\alpha} = \sin\alpha$.
The mixing is controlled by the $2\times 2$ mass-squared matrix
\begin{equation}
	\mathcal{M}^2 = \left( \begin{array}{cc}
			\mathcal{M}_{11}^2 & \mathcal{M}_{12}^2 \\
			\mathcal{M}_{12}^2 & \mathcal{M}_{22}^2 \end{array} \right),
\end{equation}
where
\begin{eqnarray}
	\mathcal{M}_{11}^2 &=& 8 \lambda_1 v_{\phi}^2, \nonumber \\
	\mathcal{M}_{12}^2 &=& \frac{\sqrt{3}}{2} v_{\phi} 
	\left[ - M_1 + 4 \left(2 \lambda_2 - \lambda_5 \right) v_{\chi} \right], \nonumber \\
	\mathcal{M}_{22}^2 &=& \frac{M_1 v_{\phi}^2}{4 v_{\chi}} - 6 M_2 v_{\chi} 
	+ 8 \left( \lambda_3 + 3 \lambda_4 \right) v_{\chi}^2.
\end{eqnarray}
The mixing angle is fixed by 
\begin{eqnarray}
	\sin 2 \alpha &=&  \frac{2 \mathcal{M}^2_{12}}{m_H^2 - m_h^2},    \nonumber  \\
	\cos 2 \alpha &=&  \frac{ \mathcal{M}^2_{22} - \mathcal{M}^2_{11}  }{m_H^2 - m_h^2},    
\end{eqnarray}
and is chosen to be in the range $\alpha \in (-\pi/2, \pi/2]$, so that $\cos\alpha \geq 0$.  The masses are given by
\begin{eqnarray}
	m^2_{h,H} &=& \frac{1}{2} \left[ \mathcal{M}_{11}^2 + \mathcal{M}_{22}^2
	\mp \sqrt{\left( \mathcal{M}_{11}^2 - \mathcal{M}_{22}^2 \right)^2 
	+ 4 \left( \mathcal{M}_{12}^2 \right)^2} \right].
	\label{eq:hmass}
\end{eqnarray}

It is convenient to use the measured mass of the observed SM-like Higgs boson as an input parameter.  The coupling $\lambda_1$ can be eliminated in favor of this mass by inverting Eq.~(\ref{eq:hmass}):
\begin{equation}
	\lambda_1 = \frac{1}{8 v_{\phi}^2} \left[ m_h^2 
	+ \frac{\left( \mathcal{M}_{12}^2 \right)^2}{\mathcal{M}_{22}^2 - m_h^2} \right].
	\label{eq:lambda1}
\end{equation}
Note that in deriving this expression for $\lambda_1$, the distinction between $m_h$ and $m_H$ is lost.  This means that, depending on the values of $\mu_3^2$ and the other parameters, this (unique) solution for $\lambda_1$ will correspond to either the lighter or the heavier custodial singlet having a mass equal to the observed SM-like Higgs mass.

\subsection{Yukawa sector}

Fermion masses are generated through couplings to the complex doublet $\phi \equiv (\phi^+,\phi^0)$ in the same was as in the SM.  We neglect neutrino masses.  The relevant Lagrangian terms are
\begin{equation}
	\mathcal{L} \supset - \sum_{i=1}^3 \sum_{j=1}^3
	\left[ y^u_{ij} \bar u_{Ri} \tilde \phi^{\dagger} Q_{Lj}
	+ y^d_{ij} \bar d_{Ri} \phi^{\dagger} Q_{Lj} \right] 
	+ y^{\ell}_i \bar \ell_{Ri} \phi^{\dagger} L_{Li}
	+ {\rm h.c.},
\end{equation}
where $i,j$ run over the three generations and $\tilde \phi \equiv i \sigma^2 \phi^*$.  
The custodial singlets and triplet contain an admixture of $\phi$, and so couple to fermions.  The custodial fiveplet states do not couple to fermions.

The Feynman rules for neutral scalars coupling to fermion pairs are given as follows:
\begin{eqnarray}
	h \bar f f: &\quad& -i \frac{m_f}{v} \frac{\cos \alpha}{\cos \theta_H}, \qquad \qquad
	H \bar f f: \quad -i \frac{m_f}{v} \frac{\sin \alpha}{\cos \theta_H}, \nonumber \\
	H_3^0 \bar u u: &\quad& \frac{m_u}{v} \tan \theta_H \gamma_5, \qquad \qquad
	H_3^0 \bar d d: \quad -\frac{m_d}{v} \tan \theta_H \gamma_5.
\end{eqnarray}
Here $f$ denotes any charged fermion, $u$ stands for any up-type quark, and $d$ stands for any down-type quark or charged lepton.  

The Feynman rules for the vertices involving a charged scalar and two fermions are given as follows, with all particles incoming:
\begin{eqnarray}
	H_3^+ \bar u d: &\quad& -i \sqrt{2} V_{ud} \tan\theta_H
		\left( \frac{m_u}{v} P_L - \frac{m_d}{v} P_R \right), \nonumber \\
	H_3^{+*} \bar d u: &\quad& -i \sqrt{2} V_{ud}^* \tan\theta_H 
		\left( \frac{m_u}{v} P_R - \frac{m_d}{v} P_L \right), \nonumber \\
	H_3^+ \bar \nu \ell: &\quad& i \sqrt{2} \tan\theta_H \frac{m_{\ell}}{v} P_R, \nonumber \\
	H_3^{+*} \bar \ell \nu: &\quad& i \sqrt{2} \tan\theta_H \frac{m_{\ell}}{v} P_L.
\end{eqnarray}
Here $V_{ud}$ is the appropriate element of the Cabibbo-Kobayashi-Maskawa matrix and the projection operators are defined as $P_{R,L} = (1 \pm \gamma_5)/2$.  


%%%%%%%%%%%%%%%%%%%%%%%%%%%%%%%%%%%%%%%%%%%%%%%%
\section{Theoretical constraints}
\label{sec:thy}

\subsection{Tree-level unitarity}

We implement the conditions for unitarity of tree-level $2\to 2$ scalar particle scattering amplitudes computed in Refs.~\cite{Aoki:2007ah,HKL}.  These were computed by imposing $|{\rm Re}\, a_0| < 1/2$ on the eigenvalues of the zeroth partial wave amplitude coupled-channel matrix, and read
\begin{eqnarray}
	\sqrt{ \left( 6 \lambda_1 - 7 \lambda_3 - 11 \lambda_4 \right)^2 + 36 \lambda_2^2}
	+ \left| 6 \lambda_1 + 7 \lambda_3 + 11 \lambda_4 \right| &<& 4 \pi, \nonumber \\
	\sqrt{ \left( 2 \lambda_1 + \lambda_3 - 2 \lambda_4 \right)^2 + \lambda_5^2}
	+ \left| 2 \lambda_1 - \lambda_3 + 2 \lambda_4 \right| &<& 4 \pi, \nonumber \\
	\left| 2 \lambda_3 + \lambda_4 \right| &<& \pi, \nonumber \\
	\left| \lambda_2 - \lambda_5 \right| &<& 2 \pi.
	\label{eq:uni}
\end{eqnarray}

\subsection{Bounded-from-below requirement on the potential}

We implement the conditions that ensure the scalar potential is bounded from below as computed in Ref.~\cite{HKL}.  They read as follows:
\begin{eqnarray}
	\lambda_1 &>& 0, \nonumber \\
	\lambda_4 &>& \left\{ \begin{array}{l l}
		- \frac{1}{3} \lambda_3 & {\rm for} \ \lambda_3 \geq 0, \\
		- \lambda_3 & {\rm for} \ \lambda_3 < 0, \end{array} \right. \nonumber \\
	\lambda_2 &>& \left\{ \begin{array}{l l}
		\frac{1}{2} \lambda_5 - 2 \sqrt{\lambda_1 \left( \frac{1}{3} \lambda_3 + \lambda_4 \right)} &
			{\rm for} \ \lambda_5 \geq 0 \ {\rm and} \ \lambda_3 \geq 0, \\
		\omega_+(\zeta) \lambda_5 - 2 \sqrt{\lambda_1 ( \zeta \lambda_3 + \lambda_4)} &
			{\rm for} \ \lambda_5 \geq 0 \ {\rm and} \ \lambda_3 < 0, \\
		\omega_-(\zeta) \lambda_5 - 2 \sqrt{\lambda_1 (\zeta \lambda_3 + \lambda_4)} &
			{\rm for} \ \lambda_5 < 0, 
			\end{array} \right.
	\label{eq:bfbcond2}
\end{eqnarray}
where 
\begin{equation}
	\omega_{\pm}(\zeta) = \frac{1}{6}(1 - B) \pm \frac{\sqrt{2}}{3} \left[ (1 - B) \left(\frac{1}{2} + B\right)\right]^{1/2},
\end{equation}
with
\begin{equation}
	B \equiv \sqrt{\frac{3}{2}\left(\zeta - \frac{1}{3}\right)} \in [0,1].
\end{equation}
The last two conditions for $\lambda_2$ in Eq.~(\ref{eq:bfbcond2}) must be satisfied for all values of $\zeta \in \left[ \frac{1}{3}, 1 \right]$.  We implement this through a 1000-point scan over $\zeta$ in the specified range.

\subsection{Absence of deeper custodial symmetry-breaking minima}

Finally, we implement a check that the scalar potential possesses no custodial symmetry-breaking minima that are deeper than the desired custodial symmetry-preserving minimum, following the procedure described in Ref.~\cite{HKL}.  We write the scalar potential as
\begin{equation}
	V = \frac{\mu_2^2}{2} a^2 + \frac{\mu_3^2}{2} b^2 + \lambda_1 a^4 + \lambda_2 a^2 b^2
	+ \zeta \lambda_3 b^4 + \lambda_4 b^4 - \omega \lambda_5 a^2 b^2 - \sigma M_1 a^2 b 
	- \rho M_2 b^3,
\end{equation}
where $a^2 = {\rm Tr}(\Phi^{\dagger}\Phi)$ and $b^2 = {\rm Tr}(X^{\dagger}X)$ and the dimensionless coefficients $\zeta$, $\omega$, $\sigma$, and $\rho$ vary with varying triplet field configurations.  The minimum of $V$ is always traced out by the path~\cite{HKL}
\begin{eqnarray}
	\zeta &=& \frac{1}{2} \sin^4 \theta + \cos^4 \theta, \nonumber \\
	\omega &=& \frac{1}{4} \sin^2 \theta + \frac{1}{\sqrt{2}} \sin \theta \cos \theta, \nonumber \\
	\sigma &=& \frac{1}{2\sqrt{2}} \sin \theta + \frac{1}{4} \cos \theta, \nonumber \\
	\rho &=& 3 \sin^2 \theta \cos \theta,
	\label{eq:thetaparam}
\end{eqnarray}
with $\theta \in [0, 2\pi)$.  Our desired electroweak-breaking and custodial SU(2)-preserving vacuum corresponds to $\theta = \cos^{-1} (1/\sqrt{3})$.  The vacuum $\theta = \pi + \cos^{-1} (1/\sqrt{3})$ is also acceptable; it corresponds to negative $b$.  The depths of these vacua are determined by applying the minimization conditions and solving the resulting cubic and quadratic equations to determine the values of $a$ and $b$ that minimize the potential, then evaluating $V$ at this minimum.

This procedure is then repeated for other values of $\theta$ [corresponding to vacua that spontaneously break custodial SU(2)] using a 1000-point scan over $\theta \in [0, 2\pi)$.  Parameter points fail this check if any vacuum solution exists in which $V$ is lower than the value in the desired vacuum.


%%%%%%%%%%%%%%%%%%%%%%%%%%%%%%%%%%%%%%%%%%%%%%%%
\section{Indirect experimental constraints}
\label{sec:indir}

Indirect constraints from the $S$ parameter, $b \to s \gamma$, and $B_s^0 \to \mu^+ \mu^-$ are implemented in the code.  A detailed physics description is given in Ref.~\cite{indirect}.  Currently the constraint from $b \to s \gamma$ is stronger than that from $B_s^0 \to \mu^+\mu^-$, but that may change in the next several years as more data is collected at the CERN Large Hadron Collider.

%-------------------------------------------------------------------------------------------------------------------------
\subsection{$S$ parameter}

When the new physics is not light compared to $M_Z$, the $S$ parameter can be written in terms of the derivatives $\Pi^{\prime}(0) \equiv d \Pi(p^2)/d p^2 |_{p^2 = 0}$ of the gauge boson self-energies as
\begin{equation}
	S = \frac{4 s_W^2 c_W^2}{\alpha_{EM}} \left[\Pi'_{ZZ}(0)
	-\frac{c_W^2-s_W^2}{s_W c_W}\Pi'_{Z\gamma}(0)-\Pi'_{\gamma\gamma}(0)\right].
\end{equation}
The new physics contribution in the GM model, relative to the SM for a reference Higgs mass $m_h^{\rm SM}$, is~\cite{indirect}
\begin{eqnarray}
	S &=& \frac{s_W^2 c_W^2}{e^2\pi}
	\left\{-\frac{e^2}{12 s_W^2 c_W^2}\left(\log{m_3^2}+5\log{m_5^2}\right)
	+2|g_{ZhH_3^0}|^2\,f_1(m_h,m_3)\right. \nonumber\\
	&& \left.+ 2|g_{ZHH_3^0}|^2\,f_1(m_H,m_3)
	+2\left(|g_{ZH_5^0H_3^0}|^2+2|g_{ZH_5^+H_3^{+*}}|^2\right)f_1(m_5,m_3)\right. 
	\nonumber\\
	&&\left.+|g_{ZZh}|^2\left[\frac{f_1(M_Z,m_h)}{2 M_Z^2}-f_3(M_Z,m_h)\right]
	+|g_{ZZH}|^2\left[\frac{f_1(M_Z,m_H)}{2 M_Z^2}-f_3(M_Z,m_H)\right]\right. \nonumber\\
	&&\left.+|g_{ZZH_5^0}|^2\left[\frac{f_1(M_Z,m_5)}{2 M_Z^2}-f_3(M_Z,m_5)\right]\right. \nonumber\\
	&&\left.+2|g_{ZW^+H_5^{+*}}|^2\left[\frac{f_1(M_W,m_5)}{2 M_W^2}-f_3(M_W,m_5)\right]
	\right. \nonumber \\
	&&\left. -|g_{ZZh}^{\rm SM}|^2\left[\frac{f_1(M_Z,m_h^{\rm SM})}{2 M_Z^2}-f_3(M_Z,m_h^{\rm SM})\right] \right\},
	\label{eq:SGM}
\end{eqnarray}
where 
\begin{eqnarray}
	f_1(m_1,m_2) &=&
	\frac{5(m_2^6-m_1^6) + 27 (m_1^4 m_2^2-m_1^2 m_2^4) 
	+ 12 (m_1^6-3 m_1^4 m_2^2) \log m_1 
	+ 12(3 m_1^2 m_2^4-m_2^6)\log m_2}{36(m_1^2-m_2^2)^3}, \nonumber \\
	f_3(m_1,m_2) &=& \frac{m_1^4 - m_2^4 
	+ 2 m_1^2 m_2^2\left(\log m_2^2 - \log m_1^2\right)}{2 (m_1^2 - m_2^2 )^3}.
\end{eqnarray}
For numerical stability we use an expansion in $\epsilon \equiv \frac{m_2^2}{m_1^2} - 1$ when $m_1^2 \simeq m_2^2$ to within a part in $10^{-4}$, 
\begin{equation}
	f_1(m_1,m_2) \simeq \frac{1}{6} \log m_1^2 + \frac{\epsilon}{12}, \qquad \qquad
	f_3(m_1,m_2) \simeq \frac{1}{6 m_1^2} - \frac{\epsilon}{12 m_1^2}.
\end{equation}

The couplings that appear in Eq.~(\ref{eq:SGM}) are given by~\cite{HKL}
\begin{eqnarray}
	g_{ZhH_3^0} &=& -i\sqrt{\frac{2}{3}}\frac{e}{s_Wc_W}\left(s_\alpha \frac{v_\phi}{v} + \sqrt{3} c_\alpha \frac{v_\chi}{v} \right), \qquad
	g_{ZHH_3^0} = i\sqrt{\frac{2}{3}}\frac{e}{s_Wc_W}\left(c_\alpha \frac{v_\phi}{v} - \sqrt{3} s_\alpha \frac{v_\chi}{v} \right), \nonumber \\
	g_{ZH_5^0 H_3^0} &=& -i \sqrt{\frac{1}{3}} \frac{e}{s_Wc_W}\frac{v_\phi}{v}, \qquad \qquad \qquad \qquad \quad
	g_{ZH_5^+H_3^{+*}} = \frac{e}{2 s_Wc_W}\frac{v_\phi}{v}, \nonumber \\
	g_{ZZh} &=& \frac{e^2}{2 s_W^2 c_W^2} \left(c_\alpha v_\phi - \frac{8}{\sqrt{3}} s_\alpha v_\chi \right),  \qquad \qquad
	g_{ZZH} = \frac{e^2}{2 s_W^2 c_W^2} \left(s_\alpha v_\phi + \frac{8}{\sqrt{3}} c_\alpha v_\chi \right), \nonumber \\
	g_{ZZ H_5^0} &=& -\sqrt{\frac{8}{3}} \frac{e^2}{s_W^2 c_W^2} v_{\chi}, \qquad \qquad \qquad \qquad \quad
	g_{Z W^+ H_5^{+*}} = -\frac{\sqrt{2} e^2}{c_W s_W^2} v_{\chi},
\end{eqnarray}
and the SM coupling $g_{ZZh}^{\rm SM}$ is given by
\begin{equation}
	g_{ZZh}^{\rm SM} = \frac{e^2 v}{2 s_W^2 c_W^2}.
\end{equation}
We use $s_{\alpha} \equiv \sin\alpha$, $c_{\alpha} \equiv \cos\alpha$, and similarly for the sine and cosine of the weak mixing angle.

For a reference SM Higgs mass of $m_h^{\rm SM} = 125$~GeV and setting $U = 0$, the global electroweak fit yields~\cite{PDG2018}
\begin{equation}
	S_{\rm exp} = 0.02 \pm 0.07, \qquad \qquad
	T_{\rm exp} = 0.06 \pm 0.06,
\end{equation}
with a correlation $\rho_{ST} = +0.92$.  These values ({\tt MHREF}, {\tt SEXP}, {\tt DSEXP}, {\tt TEXP}, {\tt DTEXP}, and {\tt RHOST}, respectively) are hard-coded in the subroutine {\tt INITINDIR} in /src/gmindir.f.

We compute the $\chi^2$ according to
\begin{equation}
	\chi^2 =  \frac{1}{\left(1-\rho_{ST}^2\right)}\left[\frac{\left(S-S_{\rm exp}\right)^2}{\left( \Delta S_{\rm exp} \right)^2}+\frac{\left(T-T_{\rm exp}\right)^2}{\left( \Delta T_{\rm exp} \right)^2}-\frac{2 \rho_{ST} \left(S-S_{\rm exp}\right)\left(T-T_{\rm exp}\right)}{\Delta S_{\rm exp} \Delta T_{\rm exp}}\right],
\end{equation}
where $\Delta S_{\rm exp}$ and $\Delta T_{\rm exp}$ are the $1\sigma$ experimental uncertainties.

It is well known that the one-loop calculation of the $T$ parameter in the GM model yields a divergent result due to the explicit breaking of the custodial symmetry by hypercharge gauge interactions~\cite{Gunion:1990dt}.  In a proper treatment $T$ acquires a counterterm, which must be set, e.g., by specifying the energy scale at which the custodial symmetry in the scalar potential is exact.  Here we take the conservative approach of marginalizing over $T$, which amounts to setting 
\begin{equation}
	T = T_{\rm exp} + \rho_{ST} (S-S_{\rm exp})\frac{\Delta T_{\rm exp}}{\Delta S_{\rm exp}}.
\end{equation}
We set the flag ${\tt SPAROK} = 1$ if the GM prediction for the $S$ parameter yields $\chi^2 \leq 4$, and ${\tt SPAROK} = 0$ otherwise.


%------------------------------------------------------------------------------
\subsection{$b \to s \gamma$}

The current world average experimental measurement of ${\rm BR}(\bar B \to X_s \gamma)$, for a photon energy $E_{\gamma} > 1.6$~GeV, is~\cite{Beringer:1900zz}
\begin{equation}
	{\rm BR}(\bar B \to X_s \gamma)_{\rm exp} = (3.55 \pm 0.24 \pm 0.09) \times 10^{-4}.
\end{equation}
To evaluate the constraint from this observable, we calculated the GM model predictions for a grid of $(m_3, v_{\chi})$ values by adapting the implementation for the Type-I 2HDM in the code SuperIso v3.3~\cite{SuperIso} (which makes use of the code 2HDMC v1.6.4~\cite{2HDMC}).  Our choice of input parameters yields a prediction in the limit $v_{\chi} \to 0$ or $m_3 \to \infty$ of
\begin{equation}
	{\rm BR}(\bar B \to X_s \gamma)_{\rm SM \, limit} = (3.11 \pm 0.23) \times 10^{-4},
\end{equation}
where the theoretical uncertainty is taken from Ref.~\cite{Misiak:2006zs}.  We scale the theoretical uncertainty by the ratio ${\rm BR}(\bar B \to X_s \gamma)_{\rm GM}/{\rm BR}(\bar B \to X_s \gamma)_{\rm SM \, limit}$ before combining it in quadrature with the experimental uncertainties.

The two data files /src/bsgtight.data and /src/bsgloose.data contain two sets of points $(m_3, v_{\chi})$ corresponding to the contour at which ${\rm BR}(\bar B \to X_s \gamma)_{\rm GM} = 2.88 \times 10^{-4}$ (``tight'' constraint) and $2.48 \times 10^{-4}$ (``loose'' constraint), respectively.  These correspond to a 2$\sigma$ deviation from the experimental central value (``tight'') and a value 2$\sigma$ ``worse'' than the SM prediction (``loose'').  For further explanation, see Ref.~\cite{indirect}.  Model points are checked for consistency with these constraints by linearly interpolating the upper bound on $v_{\chi}$ to the appropriate mass $m_3$.  For $m_3 < 10$~GeV the limit on $v_{\chi}$ for $m_3 = 10$~GeV is used, and for $m_3 > 1000$~GeV the limit on $v_{\chi}$ for $m_3 = 1000$~GeV is used.  (This latter limiting value falls outside the parameter range allowed by theoretical constraints, and so is irrelevant in practice.)

We set the flag {\tt BSGAMTIGHTOK} $=1$ if the GM prediction for ${\rm BR}(\bar B \to X_s \gamma)$ satisfies the ``tight'' 2$\sigma$ constraint, and {\tt BSGAMTIGHTOK} $=0$ otherwise.
Similarly, we set the flag {\tt BSGAMLOOSEOK} $=1$ if the GM prediction for ${\rm BR}(\bar B \to X_s \gamma)$ satisfies the ``loose'' 2$\sigma$ constraint, and {\tt BSGAMLOOSEOK} $=0$ otherwise.



%------------------------------------------------------------------------------
\subsection{$B_s^0 \to \mu^+ \mu^-$}

The time-averaged branching ratio for $B_s^0 \to \mu^+ \mu^-$, normalized to its Standard Model value, is given to an excellent approximation by the ratio of $Z$-penguin contributions~\cite{indirect,Li:2014fea}
\begin{equation}
	{\tt RBSMM} \equiv 
	\frac{\overline {\rm BR}(B_s^0 \to \mu^+ \mu^-)}{\overline {\rm BR}(B_s^0 \to \mu^+ \mu^-)_{\rm SM}}
	\simeq \left| \frac{C_{10}^{\rm SM} + C_{10}^{\rm GM}}{C_{10}^{\rm SM}} \right|^2,
\end{equation}
where~\cite{Li:2014fea}
\begin{equation}
	C_{10}^{\rm SM} = -0.9380 \left[ \frac{M_t}{173.1~{\rm GeV}} \right]^{1.53}
	\left[ \frac{\alpha_s(M_Z)}{0.1184} \right]^{-0.09}
\end{equation}
and~\cite{indirect,Li:2014fea}
\begin{equation}
	C_{10}^{\rm GM} = C_{10}^{\rm SM} + \tan^2 \theta_H \frac{x_{tW}}{8}
	\left[ \frac{x_{t3}}{1-x_{t3}} + \frac{x_{t3} \log x_{t3}}{(1-x_{t3})^2} \right],
\end{equation}
with $x_{tW} = \overline m_{t}^2(M_t)/M_W^2$ and $x_{t3} = \overline m_t^2(M_t)/m_3^2$.\footnote{The calculation of the $\overline{\rm MS}$ running top quark mass $\overline m_t(\mu)$ is described in Sec.~\ref{sec:Hff}.  $M_t$ is the pole mass.}  For numerical stability we use an expansion in $\delta \equiv x_{t3}-1$ when $x_{t3} \simeq 1$ to within a part in $10^{-4}$, 
\begin{equation}
	\left[ \frac{x_{t3}}{1-x_{t3}} + \frac{x_{t3} \log x_{t3}}{(1-x_{t3})^2} \right] \simeq -\frac{1}{2} - \frac{\delta}{6}
	\qquad \qquad (\delta \equiv x_{t3} -1 \to 0).
\end{equation}

The corresponding SM prediction and its uncertainty are~\cite{Li:2014fea}
\begin{equation}
	\overline {\rm BR}(B_s^0 \to \mu^+ \mu^-)_{\rm SM} = (3.67 \pm 0.25) \times 10^{-9}
	\left|\left[ \frac{M_t}{173.1~{\rm GeV}} \right]^{1.53}
	\left[ \frac{\alpha_s(M_Z)}{0.1184} \right]^{-0.09} \right|^2.
\end{equation}
We calculate the prediction in the GM model by scaling this prediction and its uncertainty by {\tt RBSMM}.

The current world average experimental value (from CMS and LHCb) is~\cite{bsmmexp}
\begin{equation}
	\overline {\rm BR}(B_s^0 \to \mu^+\mu^-)_{\rm expt} = (2.9 \pm 0.7) \times 10^{-9}.
\end{equation}
The experimental central value ({\tt BMMEXP}) and its uncertainty ({\tt DBMMEXP}) are hard-coded in the subroutine {\tt INITINDIR} in /src/gmindir.f.

Combining the theoretical and experimental uncertainties in quadrature, this measured value is about $1\sigma$ below the SM prediction.  The GM prediction is always higher than the SM prediction (in worse agreement with experiment) and depends only on the parameters $m_3$ and $\tan\theta_H$.  

We set the flag ${\tt BSMMOK} = 1$ if the GM prediction for $\overline {\rm BR}(B_s^0 \to \mu^+\mu^-)$ is within $2\sigma$ of the experimental value, and ${\tt BSMMOK} = 0$ otherwise.



%%%%%%%%%%%%%%%%%%%%%%%%%%%%%%%%%%%%%%%%%%%%%%
\section{Decays}
\label{sec:decays}

Starting from the tree-level masses and couplings, the code calculates the decay widths of the Higgs bosons into various final states.  At tree level the Higgs bosons can decay into pairs of fermions, pairs of massive gauge bosons, a gauge boson and a lighter Higgs boson, and two lighter Higgs bosons.  Decays of the neutral Higgs bosons into $gg$, $\gamma\gamma$, and $Z\gamma$ are induced at one loop.

\subsection{$H \to f \bar f^{\prime}$}
\label{sec:Hff}

The custodial singlet states $h$ and $H$ and the custodial triplet states $H_3^0$ and $H_3^{\pm}$ can decay to pairs of fermions.  The custodial fiveplet states do not couple to fermions.

The Feynman rule for a scalar coupling to $f \bar f^{\prime}$ is parameterized as $i (g^S + g^P \gamma_5)$, where $g^S$ is the scalar part and $g^P$ is the pseudoscalar part.  $g^S$ and $g^P$ can be simultaneously nonzero only for charged Higgs couplings to fermions.

The decay width to fermions is given by (the number of colors $N_c = 3$ for quarks and 1 for leptons)
\begin{equation}
	\Gamma(H \to f \bar f^{\prime}) = \frac{N_c m_H}{8 \pi}
	\left\{ \left[ 1 - (x_1 + x_2)^2 \right] |g^S|^2 + \left[ 1 - (x_1 - x_2)^2 \right] |g^P|^2 \right\}
	\lambda^{1/2}(x_1^2, x_2^2),
\end{equation}
where $x_1 = m_f/m_H$, $x_2 = m_{f^{\prime}}/m_H$, and the kinematic function $\lambda$ is given by
\begin{equation}
	\lambda(x,y) = (1 - x - y)^2 - 4xy.
	\label{eq:lambda}
\end{equation}

For scalar decays to quarks, we incorporate the QCD corrections as follows.
First, we incorporate the leading QCD corrections by replacing $m_q \to \overline m_q(M_H)$ in the Yukawa couplings $g^S$ and $g^P$, where $\overline m_q(M_H)$ is the $\overline{\rm MS}$ running quark mass evaluated at the scale of the parent Higgs particle's mass.  We compute the running quark masses using~\cite{Djouadi:1995gt}
\begin{equation}
	\overline m_q(\mu) = \overline m_q(M_q) \frac{c[\alpha_s(\mu)/\pi]}{c[\alpha_s(M_q)/\pi]},
\end{equation}
where
\begin{eqnarray}
	c(x) &=& \left( \frac{25}{6}x \right)^{12/25} (1 + 1.014 x + 1.389 x^2), 
		\qquad M_c < \mu < M_b \nonumber \\
	c(x) &=& \left( \frac{23}{6} x \right)^{12/23} (1 + 1.175 x + 1.501 x^2),
		\qquad M_b < \mu.
\end{eqnarray}
The running strong coupling constant is computed using~\cite{Djouadi:1995gt}
\begin{equation}
	\alpha_s^{(N_f)}(\mu) = \frac{12 \pi}{(33 - 2N_f) \log(\mu^2/\Lambda_{N_f}^2)}
		\left[ 1 - 6 \frac{(153 - 19N_f)}{(33 - 2N_f)^2} 
		\frac{\log \log(\mu^2/\Lambda_{N_f}^2)}{\log(\mu^2/\Lambda_{N_f}^2)} \right].
\end{equation}
We implement matching at the bottom quark threshold by requiring continuity of $\alpha_s$.
Above the top threshold we continue to use the five-flavor scheme for consistency with HDECAY~\cite{Djouadi:1997yw}.

Second, for decays of neutral CP-even scalars to $b \bar b$ or $c \bar c$ we incorporate the finite QCD corrections by multiplying the partial width given above by the factor~\cite{Djouadi:1995gt}
\begin{equation}
	\left[ \Delta_{QCD} + \Delta_t \right],
\end{equation}
where
\begin{eqnarray}
	\Delta_{QCD} &=& 1 + 5.67 \frac{\alpha_s(M_H)}{\pi}
		+ (35.94 - 1.36 N_f) \left( \frac{\alpha_s(M_H)}{\pi} \right)^2, \nonumber \\
	\Delta_t &=& \left( \frac{\alpha_s(M_H)}{\pi} \right)^2 
		\left[ 1.57 - \frac{2}{3} \log(M_H^2/M_t^2) 
		+ \frac{1}{9} \log^2 (\overline m_q^2(M_H)/M_H^2) \right].
\end{eqnarray}

The relevant SM inputs to GMCALC are 
\begin{equation}
	{\tt ALSMZ} = \alpha_s(M_Z), \qquad {\tt MTPOLE} = M_t, \qquad
	{\tt MBMB} = \overline m_b(m_b), \qquad {\tt MCMC} = \overline m_c(m_c).
\end{equation}
The values are set in {\tt INITIALIZE\_SM} in /src/gminit.f.
The $b$ and $c$ quark pole masses, and the running top quark mass, are obtained using the $\mathcal{O}(\alpha_s)$ relation~\cite{Djouadi:1995gt}
\begin{equation}
	\overline m_q(M_q) = M_q / [1 + 4 \alpha_s / 3 \pi].
\end{equation}


\subsection{$H \to V_1^* V_2^*$}

The custodial singlet states $h$ and $H$, as well as the neutral custodial fiveplet state $H_5^0$, can decay to $W^+W^-$ and $ZZ$.  The charged custodial fiveplet state $H_5^+$ can decay to $W^+Z$.  The doubly-charged custodial fiveplet state $H_5^{++}$ can decay to $W^+W^+$.  The custodial triplet states do not couple to pairs of massive vector bosons.

The Feynman rule for a scalar coupling to massive vector bosons $V_1^{\mu} V_2^{\nu}$ is parameterized as $i g_{H_iV_1V_2} g^{\mu\nu}$.

We compute the widths for $H_i \to V_1 V_2$, allowing both vector bosons to be off-shell, using~\cite{Romao:1998sr,Contino:2014aaa}
\begin{eqnarray}
	\Gamma(H_i\rightarrow V_1^*V_2^*) &=& \frac{1}{\pi^2}\int_0^{m_{H_i}^2} dQ_1^2 
	\int_0^{(m_{H_i} - Q_1)^2} dQ_2^2 \nonumber \\
	&& \quad \times
	\frac{Q_1^2 \, \Gamma_{V_1} / M_{V_1}}{(Q_1^2-M_{V_1}^2)^2+M_{V_1}^2\Gamma_{V_1}^2} 
	\frac{Q_2^2 \, \Gamma_{V_2}/M_{V_2}}{(Q_2^2-M_{V_2}^2)^2+M_{V_2}^2\Gamma_{V_2}^2}
	\Gamma^{H_iV_1V_2}(Q_1^2,Q_2^2)\,,
	\label{eq:V*V*int}
\end{eqnarray}
where $\Gamma_{V_i}$ is the measured total width of gauge boson $V_i$, $Q_i^2$ is the square of the four-momentum of $V_i$, and $\Gamma^{H_iV_1V_2}(Q_1^2,Q_2^2)$ is the on-shell decay width for $H_i\rightarrow V_1 V_2$ with the squared-masses of the gauge bosons $V_1$ and $V_2$ replaced by $Q_1^2$ and $Q_2^2$.  This on-shell width is given by
\begin{equation}
	\Gamma^{H_i V_1 V_2} (Q_1^2, Q_2^2)
	= S_V \frac{|g_{H_iV_1V_2}|^2 m_{H_i}^3}{64 \pi Q_1^2 Q_2^2}
	\left[ 1 - 2 k_1 - 2 k_2 + 10 k_1 k_2 + k_1^2 + k_2^2 \right]
	\lambda^{1/2}(k_1, k_2),
\end{equation}
where $k_1 = Q_1^2/m_{H_i}^2$ and $k_2 = Q_2^2/m_{H_i}^2$. Here $S_V$ is a symmetry factor given by $S_V = 1$ if $V_1$ and $V_2$ are distinct bosons (e.g., $W^+W^-$ or $Z W^+$) and $S_V = 1/2$ if $V_1$ and $V_2$ are identical bosons (e.g., $ZZ$ or $W^+W^+$).  The kinematic function $\lambda$ is defined in Eq.~(\ref{eq:lambda}).

We evaluate the doubly off-shell decay width using numerical integration.  Below threshold ($m_{H_i} \leq M_{V_1} + M_{V_2}$) we integrate Eq.~(\ref{eq:V*V*int}) using an implementation of the Vegas algorithm~\cite{Lepage:1980dq} borrowed with permission from PROSPINO~\cite{Beenakker:1996ed}.  Above threshold ($m_{H_i} > M_{V_1} + M_{V_2}$) we use 16-point Gauss-Legendre integration
after making the change of variables 
\begin{equation}
	\rho_i = \frac{1}{\pi} \tan^{-1}\left[ \frac{Q_i^2 - M_{V_i}^2}{M_{V_i} \Gamma_{V_i}} \right]
\end{equation}
to flatten the Breit-Wigners.  The number of integration points is optimized for efficiency above and below the $H_i \to V_1 V_2$ threshold, while keeping the numerical precision within 1\% of the value computed by HDECAY~6.42~\cite{Djouadi:1997yw}.

We have not taken into account the interference effects in same-flavor decays due to crossed diagrams.



\subsection{$H_1 \to V H_2$}

The custodial singlet states $h$ and $H$ can decay to a vector boson plus a custodial triplet scalar.  The custodial triplet states $H_3^0$ and $H_3^{\pm}$ can decay to a vector boson plus a custodial singlet state, or to a vector boson plus a custodial fiveplet state.  The custodial fiveplet states $H_5^0$, $H_5^{\pm}$, and $H_5^{\pm \pm}$ can decay to a vector boson plus a custodial triplet state.  

The Feynman rule for the $H_1 H_2^* V^*_{\mu}$ coupling (all particles and momenta incoming) is parameterized as $i g_{V^*H_1H_2^*} (p_1 - p_2)_{\mu}$, where $p_1$ ($p_2$) is the incoming momentum of the scalar $H_1$ ($H_2^*$).

The on-shell two-body decay width into one vector and one lighter scalar is given by
\begin{equation}
	\Gamma(H_1 \to V H_2) = \frac{|g_{V^*H_1H_2^*}|^2 M_V^2}{16 \pi m_{H_1}}
	\lambda\left( \frac{m_{H_1}^2}{M_V^2}, \frac{m_{H_2}^2}{M_V^2} \right)
	\lambda^{1/2} \left( \frac{M_V^2}{m_{H_1}^2}, \frac{m_{H_2}^2}{m_{H_1}^2} \right).
\end{equation}
Here $V$ denotes one of the gauge bosons $Z$, $W^+$, or $W^-$, such that the decays $H^0 \to W^+ H^-$ and $H^0 \to W^- H^+$ are distinct.

We also implement $H_1 \to V^* H_2$ decays (with the gauge boson off-shell) when the $H_1$ mass is below threshold for the on-shell two-body decay.  Following Ref.~\cite{Djouadi:1995gv},
\begin{equation}
	\Gamma(H_1 \to V^* H_2) = \delta_V \frac{3 |g_{V^* H_1 H_2^*} |^2 M_V^2 m_{H_1}}
	{16 \pi^3 v^2} G_{H_2 V},
\end{equation}
where again $V$ denotes one of the gauge bosons $Z$, $W^+$, or $W^-$, such that the decays $H^0 \to W^+ H^-$ and $H^0 \to W^- H^+$ are distinct.  $\delta_W$ and $\delta_Z$ are given by\footnote{We absorb the factor of $c_W^4$ that appears in the denominator of $\delta_Z$ in Eq.~(36) of Ref.~\cite{Djouadi:1995gv} into the coupling.  We also separate out a symmetry factor of $2$ from $\delta_W$ for convenience.}
\begin{equation}
	\delta_W = \frac{3}{2}, \qquad \qquad
	\delta_Z = 3 \left( \frac{7}{12} - \frac{10}{9} s_W^2 + \frac{40}{27} s_W^4 \right).
\end{equation}
The kinematic function $G_{ij}$ is defined as follows (here we fix a typing error in Ref.~\cite{Djouadi:1995gv} as pointed out in Ref.~\cite{Akeroyd:1998dt}: the last term is $+2 \lambda_{ij}/k_j$ rather than $-2 \lambda_{ij}/k_j$):
\begin{eqnarray}
	G_{ij} &=& \frac{1}{4} \left\{ 2 (-1 + k_j - k_i) \sqrt{\lambda_{ij}} 
	\left[ \frac{\pi}{2} + \arctan \left( \frac{k_j (1 - k_j + k_i) - \lambda_{ij}}{(1 - k_i) \sqrt{\lambda_{ij}}} \right) \right] \right. \nonumber \\
	&& \qquad \left. + (\lambda_{ij} - 2 k_i) \log k_i 
	+ \frac{1}{3} (1 - k_i) \left[ 5 (1 + k_i) - 4 k_j + \frac{2 \lambda_{ij}}{k_j} \right] \right\},
\end{eqnarray}
where $k_i \equiv k_{H_2} = m_{H_2}^2/m_{H_1}^2$, $k_j \equiv k_V = M_V^2/m_{H_1}^2$, and 
\begin{equation}
	\lambda_{ij} = -1 + 2 k_i + 2 k_j - (k_i - k_j)^2.
\end{equation}

\subsection{$H_1 \to H_2 H_3$}

The custodial singlet states $h$ and $H$ can decay into a pair of custodial triplet states or a pair of custodial fiveplet states.  Furthermore $H$ can decay into $hh$.  The custodial fiveplet states $H_5^0$, $H_5^{\pm}$, and $H_5^{\pm \pm}$ can decay into a pair of custodial triplet states.  The custodial triplet states cannot decay into pairs of scalars due to a combination of custodial SU(2) invariance and Bose symmetry.

The Feynman rule for the $H_1 H_2^* H_3^*$ coupling (all particles incoming) is parameterized as $-i g_{123}$.

The decay width for $H_1$ into two lighter scalars $H_2 H_3$ is
\begin{equation}
	\Gamma(H_1 \to H_2 H_3) = S_H \frac{|g_{123}|^2}{16 \pi m_{H_1}}
	\lambda^{1/2}(X_2, X_3),
\end{equation}
where $X_2 = m_{H_2}^2/m_{H_1}^2$ and $X_3 = m_{H_3}^2/m_{H_1}^2$, and $S_H$ is a symmetry factor given by $S_H = 1$ if $H_2$ and $H_3$ are distinct bosons and $S_H = 1/2$ if $H_2$ and $H_3$ are identical bosons.

\subsection{$H \to \gamma\gamma$}

Neutral scalar decays into two photons proceed through a loop of charged particles.  The width is given by~\cite{HHG}
\begin{equation}
	\Gamma(H \to \gamma\gamma) = \frac{\alpha_{EM}^2 m_H^3}{256 \pi^3 v^2}
	| \mathcal{A}_H^{\gamma\gamma} |^2,
\end{equation}
where $\alpha_{EM}$ is the electromagnetic fine-structure constant, $v = (\sqrt{2} G_F)^{-1/2} \simeq 246$~GeV is the SM Higgs vacuum expectation value, and $\mathcal{A}_H^{\gamma\gamma}$ represents the sum of the loop amplitudes for initial particle $H$.

For an initial scalar ($S = h$, $H$, or $H_5^0$), the amplitude receives contributions from fermions, $W$ bosons, and charged Higgs bosons ($H_3^+$, $H_5^+$, and $H_5^{++}$) in the loop, and is given by
\begin{equation}
	\mathcal{A}_S^{\gamma\gamma} 
	= \kappa_f^S \sum_f N_{cf} Q_f^2 F_{1/2}(\tau_f)
	+ \kappa_W^S F_1(\tau_W)
	+ \sum_s \beta_s^S Q_s^2 F_0(\tau_s).
\end{equation}
For the fermion loops, $N_{cf}$ and $Q_f$ are the number of colors and electric charge in units of $e$, respectively, for fermion $f$, and $\kappa_f^S$ is the scaling factor for the coupling of $S$ to fermions relative to the corresponding coupling of the SM Higgs boson, defined in such a way that the Feynman rule for the $S f \bar f$ coupling is $-i (m_f/v) \kappa^S_f$.  The custodial fiveplet does not couple to fermions, so $\kappa_f^{H_5^0} = 0$.  In the code we include only the top quark loop.

For the $W$ loop, $\kappa_W^S$ is the scaling factor for the coupling of $S$ to $W$ pairs relative to the corresponding coupling of the SM Higgs boson, defined so that the $S W^+_{\mu} W^-_{\nu}$ Feynman rule is $i \kappa_W^S (2 M_W^2/v) g_{\mu\nu}$.

For the scalar loops, the sum over $s$ runs over all electrically charged scalars in the GM model ($H_3^+$, $H_5^+$, and $H_5^{++}$).  $Q_s$ is the electric charge of scalar $s$ in units of $e$, and $\beta_s^S = g_{Sss^*} v/2 m_s^2$.  The coupling $g_{Sss^*}$ is defined in such a way that the corresponding interaction Lagrangian term is $\mathcal{L} \supset - g_{Sss^*} S s s^*$.

The loop factors are given in terms of the usual functions~\cite{HHG}, 
\begin{eqnarray}
	F_1(\tau) &=& 2+3\tau+3\tau(2-\tau) f(\tau), \nonumber \\
	F_{1/2}(\tau) &=& -2\tau[1+(1-\tau) f(\tau)], \nonumber \\
	F_0(\tau) &=& \tau[1-\tau f(\tau)],
\end{eqnarray}
where
\begin{equation}
	f(\tau) = \left\{ \begin{array}{l l}
	\left[\sin^{-1} \left(\sqrt{\frac{1}{\tau}}\right) \right]^2 & \quad  {\rm if} \ \tau \geq 1, \\
	-\frac{1}{4}\left[ \log \left(\frac{\eta_+}{\eta_-}\right) - i \pi \right]^2 & \quad  {\rm if} \ \tau < 1, \\
	\end{array} \right.
	\label{feq}
\end{equation}
with $\eta_{\pm} = 1 \pm \sqrt{1-\tau}$.  The argument is $\tau_i \equiv 4 m_i^2/ m_h^2$. 

For an initial pseudoscalar ($A = H_3^0$), the amplitude receives contributions only from fermions in the loop, and is given by
\begin{equation}
	\mathcal{A}_A^{\gamma\gamma} = \kappa_f^A \sum_f N_{cf} Q_f^2 F_{1/2}^A(\tau_f)
\end{equation}
where the Feynman rule for the $A f \bar f$ coupling is defined as $- (m_f/v) \kappa_f^A \gamma_5$ and the loop function is
\begin{equation}
	F_{1/2}^A(\tau) = -2 \tau f(\tau).
\end{equation}
In the code we include only the top quark loop.

\subsection{$H \to gg$}

Neutral scalar decays to two gluons proceed through a loop of colored particles.  In the GM model, the only colored particles are the SM quarks.  Therefore this decay occurs only for $h$, $H$, and $H_3^0$ (the custodial fiveplet does not couple to fermions).

The width is given by~\cite{HHG}
\begin{equation}
	\Gamma(H \to gg) = \frac{\alpha_s^2 m_H^3}{128 \pi^3 v^2} |\mathcal{A}_H^{gg}|^2,
\end{equation}
where $\mathcal{A}_H^{gg}$ represents the sum of the loop amplitudes for initial particle $H$.

For an initial scalar ($S = h$ or $H$), the amplitude is
\begin{equation}
	\mathcal{A}_S^{gg} = \kappa_f^S \sum_f F_{1/2}(\tau_f).
\end{equation}

For an initial pseudoscalar ($A = H_3^0$), the amplitude is
\begin{equation}
	\mathcal{A}_A^{gg} = \kappa_f^A \sum_f F_{1/2}^A(\tau_f).
\end{equation}

We incorporate the QCD corrections as follows.  First, we evaluate $\alpha_s$ in the leading-order amplitude at the scale of the parent particle's mass.  Second, for the decays of CP-even neutral scalars, we multiply the leading order amplitude by the factor~\cite{Djouadi:1995gt}
\begin{equation}
	\left[ 1 + E^{N_f} \alpha_s^{(N_f)}/\pi \right],
\end{equation}
where
\begin{equation}
	E^{N_f} = \frac{95}{4} - \frac{7}{6} N_f + \frac{33 - 2N_f}{6} \log(\mu^2/M_H^2),
\end{equation}
and we use $N_f = 5$ throughout, consistent with {\tt NF-GG} $= 5$ in HDECAY~\cite{Djouadi:1997yw}.

In the code we include only the top quark loop.

%%%%%%%%%%%%%%%%%%%%%%%%%%%%%%%%%%%%%%%%%%%%%
\subsection{$H \to Z \gamma$}

Neutral scalar decays to $Z$ plus a photon proceed through a loop of charged particles.  The width is given by~\cite{HHG}
\begin{equation}
	\Gamma(H \to Z \gamma) = \frac{\alpha_{EM}^2 m_H^3}{128 \pi^3 v^2}
	| \mathcal{A}_H^{Z\gamma} |^2 \left( 1 - \frac{M_Z^2}{m_H^2} \right)^3,
\end{equation}
where $\mathcal{A}_H^{Z\gamma}$ represents the sum of the loop amplitudes for initial particle $H$.

For an initial custodial-singlet scalar ($S = h$, $H$; see below for $H_5^0$), the amplitude is
\begin{equation}
	\mathcal{A}_S^{Z\gamma} = \kappa_f^S A_f + \kappa_V^S A_W + \frac{v}{2} A_s,
\end{equation}
where the contributions from fermions, $W$ bosons, and scalars are given by~\cite{HHG}
\begin{eqnarray}
	A_f &=& \sum_f N_{cf} 
	\frac{-2 Q_f \left(T^{3L}_f - 2 Q_f \sin^2\theta_W\right)}{\sin\theta_W\cos\theta_W}
	\left[ I_1(\tau_f,\lambda_f)-I_2(\tau_f,\lambda_f) \right], \nonumber \\
	A_W &=& -\cot\theta_W\left\{4\left(3-\tan^2\theta_W\right) I_2\left(\tau_W,\lambda_W\right)+\left[\left(1+\frac{2}{\tau_W}\right)\tan^2\theta_W-\left(5+\frac{2}{\tau_W}\right)\right] I_1\left(\tau_W,\lambda_W\right)\right\}, \nonumber \\
	A_s &=& \sum_s 2 \frac{g_{Sss^*}\,C_{Zss^*}\,Q_s}{m_s^2}
	I_1\left( \tau_s, \lambda_s \right).
	\label{eq:Zgaamps}
\end{eqnarray}
Here $T^{3L}_f = \pm 1/2$ is the third component of isospin for the left-handed fermion $f$.  In the code we include only the top quark loop.  The scalar amplitude depends on the coupling $C_{Zss^*} \equiv g_{Zss^*}/e$ of the scalar to the $Z$ boson, defined in  such a way that the corresponding coupling of the scalar to the photon is $C_{\gamma s s^*} \equiv g_{\gamma s s^*}/e = Q_s$.  The sum over scalars in $A_s$ runs over $H_3^+$, $H_5^+$, and $H_5^{++}$.

The loop factors are given in terms of the functions~\cite{HHG}
\begin{eqnarray}
	 I_1(a,b) &=& \frac{ab}{2(a-b)} + \frac{a^2b^2}{2(a-b)^2} \left[f(a) - f(b)\right]
	 + \frac{a^2b}{(a-b)^2} \left[g(a) - g(b)\right], \nonumber \\
	 I_2(a,b) &=& -\frac{ab}{2(a-b)} \left[f(a) - f(b)\right],
	 \label{eq:I1I2}
\end{eqnarray}
where the function $f(\tau)$ was given in Eq.~(\ref{feq}) and
\begin{equation}
	g(\tau) = \left\{ \begin{array}{l l}
	\sqrt{\tau-1} \sin^{-1} \left(\sqrt{\frac{1}{\tau}}\right) & \quad  {\rm if} \ \tau \geq 1, \\
	\frac{1}{2} \sqrt{1-\tau} \left[ \log \left(\frac{\eta_+}{\eta_-}\right) - i \pi \right] 
		& \quad  {\rm if} \ \tau < 1,
	\end{array} \right.
	\label{geq}
\end{equation}
with $\eta_{\pm}$ defined as for $f(\tau)$.  The arguments of the functions are $\tau_i \equiv 4 m_i^2/m_h^2$ as before and $\lambda_i \equiv 4 m_i^2/M_Z^2$.

For an initial pseudoscalar ($A = H_3^0$), the amplitude is
\begin{equation}
	\mathcal{A}_A^{Z\gamma} = \kappa_f^A \sum_f N_{cf} 
	\frac{-2 Q_f \left(T^{3L}_f - 2 Q_f \sin^2\theta_W\right)}{\sin\theta_W\cos\theta_W}
	\left[ -I_2(\tau_f,\lambda_f) \right].
\end{equation}
Again in the code we include only the top quark loop.

For an initial custodial fiveplet scalar ($S = H_5^0$), the amplitude is~\cite{loops}  \\
\begin{equation}
	\mathcal{A}_S^{Z\gamma} = \kappa_W^S A_W + \frac{v}{2} A_s 
	- \frac{2 \pi v}{\alpha_{\rm em}} \left[ 2 A_{WH_5H_5} + 2 A_{H_5WW} \right],
\end{equation}
where $A_W$ and $A_s$ were given in Eq.~(\ref{eq:Zgaamps}).  The amplitudes from loops involving a $W$ boson and an $H_5^{\pm}$ scalar are~\cite{loops}
\begin{eqnarray}
	A_{WH_5H_5} &=& 2 \alpha_{\rm em}^2 
	\sqrt{\frac{3}{2}} \frac{v_{\chi}}{\sin^3 \theta_W \cos\theta_W}
	\left[ C_{12} + C_{22} + 2 C_1 + 3 C_2 + 2 C_0 \right]
	(M_Z^2, 0, m_5^2; M_W^2, m_5^2, m_5^2), \nonumber \\
	A_{H_5WW} &=& \alpha_{\rm em}^2 
	\sqrt{\frac{3}{2}} \frac{v_{\chi}}{\sin^3 \theta_W \cos\theta_W}
	\left[ -2 C_{12} - 2 C_{22} + 4 C_1 + 2 C_2 \right]
	(M_Z^2, 0, m_5^2; m_5^2, M_W^2, M_W^2),
\end{eqnarray}
where $C_{i}$, $C_{ij}$ are LoopTools functions~\cite{Hahn:1998yk}.  

%%%%%%%%%%%%%%%%%%%%%%%%%%%%%%%%%%%%%%%%%%%%%
\subsection{$H \to W \gamma$}

Singly-charged scalar decays to $W$ plus a photon proceed through a loop.  The width is given by~\cite{loops}
\begin{equation}
	\Gamma(H^+ \to W^+ \gamma) = \frac{m_H^3}{32 \pi} \left[1 - \frac{M_W^2}{m_H^2} \right]^3 
	\left( |\mathcal{A}_H|^2 + |\tilde \mathcal{A}_H|^2 \right),
\end{equation}
where $\mathcal{A}_H$ and $\tilde \mathcal{A}_H$ represent the CP-even and CP-odd parts of the sum of loop amplitudes for initial particle $H$. 

For an initial custodial-fiveplet scalar $H = H_5^+$, the CP-odd part $\tilde \mathcal{A}_H$ is zero, and the CP-even part of the amplitude is
\begin{equation}
	\mathcal{A}_H = \sum_{s_1s_2} A_{s_1s_2s_2} + \sum_{Xs} A_{Xss} 
		+ \sum_{sX} A_{sXX} + A_{ZWW}.
\end{equation}

The scalar loop contribution is given by~\cite{loops,HHG}
\begin{equation}
	A_{s_1s_2s_2} = - \frac{\alpha_{\rm em}}{\pi} Q_{s_2} C_{H_5^+ s_1^* s_2} C_{W^- s_1 s_2^*} \frac{1}{4 m_s^2} I_1(\tau_s, \lambda_s),
\end{equation}
where $I_1(a,b)$ was given in Eq.~(\ref{eq:I1I2}), and the sum runs over $s_1s_2 = H_3^0 H_3^-$, $H_5^0 H_5^-$, $H_5^- H_5^{--}$, and $H_5^{++} H_5^+$, with masses $m_{s_1} = m_{s_2} \equiv m_s$.  The products of couplings that appear are
\begin{eqnarray}
	Q_{H_3^-} C_{H_5^+ H_3^0 H_3^-} C_{W^- H_3^0 H_3^+} &=& - \frac{1}{\sqrt{2} v \sin\theta_W} 
	\left[ 2(\lambda_3 - 2 \lambda_5) v_{\phi}^2 v_{\chi} - 8 \lambda_5 v_{\chi}^3 
	+ 4 M_1 v_{\chi}^2 + 3 M_2 v_{\phi}^2 \right], \\
	Q_{H_5^-} C_{H_5^+ H_5^0 H_5^-} C_{W^- H_5^0 H_5^-} &=& - \frac{3}{\sqrt{2} \sin\theta_W} (2 \lambda_3 v_{\chi} - M_2), \\
	Q_{H_5^{--}} C_{H_5^+ H_5^+ H_5^{--}} C_{W^- H_5^- H_5^{++}} &=& -\frac{6 \sqrt{2}}{\sin\theta_W} (2 \lambda_3 v_{\chi} - M_2), \\
	Q_{H_5^+} C_{H_5^+ H_5^{--} H_5^+} C_{W^- H_5^{++} H_5^-} &=& -\frac{3 \sqrt{2}}{\sin\theta_W} (2 \lambda_3 v_{\chi} - M_2).
\end{eqnarray}

The remaining pieces of the amplitude are given by~\cite{loops}
\begin{eqnarray}
	A_{Xss} &=& 2 \alpha_{\rm em}^2 Q_s C_{X^* H_5^+ s} C_{s^* X W^-}
	\left[ C_{12} + C_{22} + 2 C_1 + 3 C_2 + 2 C_0 \right]
	(M_W^2, 0, m_5^2; M_X^2, m_5^2, m_5^2), \\
	A_{sXX} &=& \alpha_{\rm em}^2 Q_X C_{X H_5^+ s^*} C_{s X^* W^-}
	\left[ -2 C_{12} - 2 C_{22} + 4 C_1 + 2 C_2 \right]
	(M_W^2, 0, m_5^2; m_5^2, M_W^2, M_W^2), \\
	A_{ZWW} &=& - \frac{\alpha_{\rm em}}{2 \pi v} \sin\theta_H M_W M_Z \cot\theta_W
	\left[ (12 C_{12} + 12 C_{22} + 12 C_2 + 6 C_0) 
	+ \frac{m_5^2}{M_W^2} ( C_{12} + C_{22} + C_2 ) \right. \nonumber \\
	&& \left. + \frac{s^2_W}{c^2_W} ( C_{12} + C_{22} + 2 C_1 + 3 C_2 + 2 C_0 )
	\right]
	(M_W^2, 0, m_5^2; M_Z^2, M_W^2, M_W^2),
\end{eqnarray}
where $C_i$, $C_{ij}$ are LoopTools functions~\cite{Hahn:1998yk}.  For the vector-scalar-scalar loop $A_{Xss}$ the sum runs over $Xs = Z H_5^-$, $W^- H_5^{--}$, and the products of couplings that appear are
\begin{eqnarray}
	Q_{H_5^-} C_{ZH_5^+H_5^-} C_{H_5^+ZW^-} &=& \frac{v_{\chi}}{\sqrt{2} \sin^3\theta_W \cos^2\theta_W} (1 - 2 \sin^2\theta_W), \\
	Q_{H_5^{--}} C_{W^+H_5^+H_5^{--}} C_{H_5^{++}W^-W^-} &=& - \frac{2 \sqrt{2} v_{\chi}}{\sin^3\theta_W}.
\end{eqnarray}
For the scalar-vector-vector loop $A_{sXX}$ the sum runs over $sX = H_5^0 W^-$, $H_5^{++} W^+$, and the products of couplings that appear are
\begin{eqnarray}
	Q_{W^-} C_{W^- H_5^+ H_5^0} C_{H_5^0 W^+ W^-} &=& \frac{v_{\chi}}{\sqrt{2} \sin^3\theta_W}, \\
	Q_{W^+} C_{W^+ H_5^+ H_5^{--}} C_{H_5^{++}W^- W^-} &=& \frac{\sqrt{2} v_{\chi}}{\sin^3\theta_W}.
\end{eqnarray}

For an initial custodial-triplet scalar $H = H_3^+$, the CP-odd part of the amplitude comes from loops involving top and bottom quarks and is given by~\cite{loops}
\begin{eqnarray}
	\tilde \mathcal{A}_H &=& \frac{\alpha_{\rm em} N_c |V_{tb}|^2}{2 \pi v \sin\theta_W} \tan\theta_H
	\left\{ Q_b \left[ -m_t^2 (C_1 + C_2 + C_0) + m_b^2 (C_1 + C_2) \right] (M_W^2, 0, m_3^2; m_t^2, m_b^2, m_b^2) \right. \nonumber \\
	&& + \left. Q_t \left[ -m_b^2 (C_1 + C_2 + C_0) + m_t^2 (C_1 + C_2) \right] (M_W^2, 0, m_3^2; m_b^2, m_t^2, m_t^2) \right\},
\end{eqnarray}
while the CP-even part of the amplitude is given by
\begin{equation}
	\mathcal{A}_H = A_f + \sum_{s_1s_2} A_{s_1s_2s_2} + \sum_{Xs} A_{Xss} 
		+ \sum_{sX} A_{sXX}.
\end{equation}
The fermion loop contribution is (we again include only the contribution from top and bottom quarks)
\begin{eqnarray}
	A_f &=& \frac{\alpha_{\rm em} N_c |V_{tb}|^2}{2 \pi v \sin\theta_W} \tan\theta_H
	\left\{ Q_b \left[ m_t^2 (2 C_{12} + 2 C_{22} + 3 C_2 + C_1 + C_0) \right. \right. \nonumber \\
	&& \qquad \qquad \qquad \qquad \left. \left. 
	- m_b^2 (2 C_{12} + 2 C_{22} + C_2 - C_1) \right] (M_W^2, 0, m_3^2; m_t^2, m_b^2, m_b^2) 
	\right. \nonumber \\
	&& \qquad \qquad \qquad \left. + Q_t \left[ -m_b^2 (2 C_{12} + 2 C_{22} + 3 C_2 + C_1 + C_0) \right. \right. \nonumber \\
	&& \qquad \qquad \qquad \qquad \left. \left.
	+ m_t^2 (2 C_{12} + 2 C_{22} + C_2 - C_1) \right] (M_W^2, 0, m_3^2; m_b^2, m_t^2, m_t^2)
	\right\}.
\end{eqnarray}
The remaining pieces of the amplitude are given by~\cite{loops}
\begin{eqnarray}
	A_{s_1s_2s_2} &=& - \frac{\alpha_{\rm em}}{\pi} 
	Q_{s_2} C_{H_3^+ s_1^* s_2} C_{W^- s_1 s_2^*}
	\left[ C_{12} + C_{22} + C_2 \right] (M_W^2, 0, m_3^2; m_{s_1}^2, m_{s_2}^2, m_{s_3}^2),
	\\
	A_{Xss} &=& 2 \alpha_{\rm em}^2 Q_s C_{X^* H_3^+ s} C_{s^* X W^-} 
	\left[ C_{12} + C_{22} + 2 C_1 + 3 C_2 + 2 C_0 \right. \nonumber \\
	&& \left. + \left( \frac{m_3^2 - m_5^2}{M_X^2} \right) (C_{12} + C_{22} + C_2) \right]
	(M_W^2, 0, m_3^2; M_X^2, m_5^2, m_5^2),
	\\
	A_{sXX} &=& \alpha_{\rm em}^2 Q_X C_{X H_3^+ s^*} C_{s X^* W^-}
	\left[ -2 C_{12} - 2 C_{22} + 4 C_1 + 2 C_2 \right. \nonumber \\
	&& \left. - 2 \left( \frac{m_3^2 - m_s^2}{M_X^2} \right) (C_{12} + C_{22} + C_2) \right]
	(M_W^2, 0, m_3^2; m_s^2, M_X^2, M_X^2),
\end{eqnarray}
where $C_i$, $C_{ij}$ are LoopTools functions~\cite{Hahn:1998yk}.  For the scalar loop $A_{s_1s_2s_2}$ the sum runs over $s_1 s_2 = h H_3^-$, $H H_3^-$, $H_5^0 H_3^-$, $H_3^0 H_5^-$, $H_5^{++} H_3^+$, $H_3^- H_5^{--}$, and the products of couplings that appear are
\begin{eqnarray}
	Q_{H_3^-} C_{H_3^+ h H_3^-} C_{W^- h H_3^+} &=& - \frac{\sqrt{2}}{3}
	\frac{(\sqrt{3} c_{\alpha} v_{\chi} + s_{\alpha} v_{\phi})}{v^3 \sin\theta_W} \nonumber\\
	&& \times \left\{ - s_{\alpha} \left[ 8 (\lambda_3 + 3 \lambda_4 + \lambda_5) v_{\phi}^2 v_{\chi} + 16 (6 \lambda_2 + \lambda_5) v_{\chi}^3 + 4 M_1 v_{\chi}^2 - 6 M_2 v_{\phi}^2 \right] \right. \nonumber \\
	&& \left. + \sqrt{3} c_{\alpha} \left[ (4 \lambda_2 - \lambda_5) v_{\phi}^3 + 
		8 (8 \lambda_1 + \lambda_5) v_{\phi} v_{\chi}^2 + 4 M_1 v_{\phi} v_{\chi} \right] 
	\right\} \\
%
	Q_{H_3^-} C_{H_3^+ H H_3^-} C_{W^- H H_3^+} &=& - \frac{\sqrt{2}}{3} 
	\frac{(\sqrt{3} s_{\alpha} v_{\chi} - c_{\alpha} v_{\phi})}{v^3 \sin\theta_W} \nonumber \\
	&& \times \left\{ c_{\alpha} \left[ 8 (\lambda_3 + 3 \lambda_4 + \lambda_5) v_{\phi}^2 v_{\chi} + 16 (6 \lambda_2 + \lambda_5) v_{\chi}^3 + 4 M_1 v_{\chi}^2 - 6 M_2 v_{\phi}^2 \right] \right. \nonumber \\
	&& \left. + \sqrt{3} s_{\alpha} \left[ (4 \lambda_2 - \lambda_5) v_{\phi}^3 + 
		8 (8 \lambda_1 + \lambda_5) v_{\phi} v_{\chi}^2 + 4 M_1 v_{\phi} v_{\chi} \right] 
	\right\}, \\
%
	Q_{H_3^-} C_{H_3^+ H_5^0 H_3^-} C_{W^- H_5^0 H_3^+} &=& \frac{v_{\phi}}{3 \sqrt{2} v^3 \sin\theta_W} \left[ 2(\lambda_3 - 2 \lambda_5) v_{\phi}^2 v_{\chi} - 8 \lambda_5 v_{\chi}^3 + 4 M_1 v_{\chi}^2 + 3 M_2 v_{\phi}^2 \right], \\
%
	Q_{H_5^-} C_{H_3^+ H_3^0 H_5^-} C_{W^- H_3^0 H_5^+} &=& -\frac{v_{\phi}}{\sqrt{2} v^3 \sin\theta_W} \left[ 2(\lambda_3 - 2 \lambda_5) v_{\phi}^2 v_{\chi} - 8 \lambda_5 v_{\chi}^3 + 4 M_1 v_{\chi}^2 + 3 M_2 v_{\phi}^2 \right], \\
%
	Q_{H_3^+} C_{H_3^+ H_5^{--} H_3^+} C_{W^- H_5^{++} H_3^-} &=& - \frac{\sqrt{2} v_{\phi}}{v^3 \sin\theta_W} \left[ 2(\lambda_3 - 2 \lambda_5) v_{\phi}^2 v_{\chi} - 8 \lambda_5 v_{\chi}^3 + 4 M_1 v_{\chi}^2 + 3 M_2 v_{\phi}^2 \right], \\
%
	Q_{H_5^{--}} C_{H_3^+ H_3^+ H_5^{--}} C_{W^- H_3^- H_5^{++}} &=& - \frac{2 \sqrt{2} v_{\phi}}{v^3 \sin\theta_W} \left[ 2(\lambda_3 - 2 \lambda_5) v_{\phi}^2 v_{\chi} - 8 \lambda_5 v_{\chi}^3 + 4 M_1 v_{\chi}^2 + 3 M_2 v_{\phi}^2 \right].
\end{eqnarray}
For the vector-scalar-scalar loop $A_{Xss}$ the sum runs over $Xs = Z H_5^-$, $W^- H_5^{--}$, and the products of couplings that appear are
\begin{eqnarray}
	Q_{H_5^-} C_{ZH_3^+H_5^-} C_{H_5^+Z W^-} &=& - \frac{v_{\phi} v_{\chi}}{\sqrt{2} v \sin^3\theta_W \cos^2\theta_W}, \\
	Q_{H_5^{--}} C_{W^+H_3^+H_5^{--}} C_{H_5^{++} W^- W^-} &=& - \frac{2 \sqrt{2} v_{\phi} v_{\chi}}{v \sin^3\theta_W}.
\end{eqnarray}
For the scalar-vector-vector loop $A_{sXX}$ the sum runs over $sX = hW^-$, $HW^-$, $H_5^0W^-$, $H_5^{++} W^+$, and the products of couplings that appear are
\begin{eqnarray}
	Q_{W^-} C_{W^- H_3^+ h} C_{h W^+ W^-} &=& \frac{1}{3 \sqrt{2} v \sin^3\theta_W} 
	(\sqrt{3} c_{\alpha} v_{\chi} + s_{\alpha} v_{\phi}) 
	(- 8 s_{\alpha} v_{\chi} + \sqrt{3} c_{\alpha} v_{\phi}), \\
	Q_{W^-} C_{W^- H_3^+ H} C_{H W^+ W^-} &=& \frac{1}{3 \sqrt{2} v \sin^3\theta_W}
	(\sqrt{3} s_{\alpha} v_{\chi} - c_{\alpha} v_{\phi})
	(8 c_{\alpha} v_{\chi} + \sqrt{3} s_{\alpha} v_{\phi}), \\
	Q_{W^-} C_{W^- H_3^+ H_5^0} C_{H_5^0 W^+ W^-} &=& - \frac{v_{\phi} v_{\chi}}{3 \sqrt{2} v \sin^3\theta_W}, \\
	Q_{W^+} C_{W^+ H_3^+ H_5^{--}} C_{H_5^{++} W^- W^-} &=& \frac{\sqrt{2} v_{\phi} v_{\chi}}{v \sin^3\theta_W}.
\end{eqnarray}


%%%%%%%%%%%%%%%%%%%%%%%%%%%%%%%%%%%%%%%%%%%%%%%%
\section{Direct constraints}
\label{sec:direct}

\subsection{Interface to HiggsBounds 5.3.0 and HiggsSignals 2.2.1}

GMCALC contains an interface to HiggsBounds and HiggsSignals to implement constraints from direct searches for Higgs bosons and from measurements of signal strengths of the 125 GeV Higgs, respectively. 

We run HiggsBounds in the effective coupling mode, with the various coupling modification factors calculated by GMCALC, and apply it to the neutral and singly-charged scalars in the GM model except for the 125~GeV Higgs boson.  HiggsSignals is used to constrain the 125~GeV Higgs boson's couplings.  For consistency with the operation of HiggsBounds, the total decay width of each neutral Higgs boson is recalculated using the coupling modification factors and SM Higgs branching ratios (determined using HiggsBounds internal functions) before being passed to HiggsBounds, as opposed to using the value computed by GMCALC.  This is required to ensure the total width is consistent with the sum of the partial widths.  We set the rate and mass uncertainties in HiggsBounds to zero. 

The calculation of coupling modification factors and other input parameters, as well as the function calls to run HiggsBounds and HiggsSignals, are implemented in the subroutine {\tt CALLHBHS}. Initializing HiggsBounds and printing the results must be done outside of this subroutine. This is demonstrated in the sample program {\tt gmhb5.f}.

\subsection{Additional search channels}

HiggsBounds~5.3.0 does not include constraints from searches for doubly-charged Higgs bosons or from Drell-Yan production of a neutral Higgs boson decaying to $\gamma\gamma$.  These processes constitute some of the strongest direct constraints on the GM model.  We implement the following processes directly in GMCALC.  

\subsubsection{VBF $H_5^{\pm\pm} \to W^{\pm} W^{\pm} \to$ like-sign dileptons}

The current most sensitive search for vector boson fusion (VBF) production of $H_5^{\pm\pm}$ with decays to $W^{\pm}W^{\pm}$ for $m_5 \geq 200$~GeV is from a CMS analysis of 137~fb$^{-1}$ of LHC Run 2 (13~TeV) data~\cite{CMS:2021wlt}.  The upper bound on $s_H$ as a function of $m_5$ assumes BR($H_5^{++} \to W^+W^+) = 1$.  We take into account the possibility that BR($H_5^{++} \to W^+W^+) < 1$ when $m_3 < m_5$ by using the fact that the signal production cross section is proportional to $s_H^2$, so that
\begin{equation}
	(s_H^{\rm limit})^2 \times {\rm BR}(H_5^{++} \to W^+W^+) = (s_H^{\rm CMS})^2,
	\label{eq:BRlt1}
\end{equation}
where $s_H^{\rm CMS}$ is the limit from Ref.~\cite{CMS:2021wlt} for BR($H_5^{++} \to W^+W^+) = 1$.  This constraint is accessed by calling the subroutine {\tt CALCH5PP}, which sets the flag {\tt H5PPOK = 1} (allowed) or {\tt 0} (excluded).

For $m_5 < 200$~GeV, VBF production of $H_5^{\pm\pm}$ with decays to $W^{\pm}W^{\pm}$ is constrained by an ATLAS measurement of the VBF like-sign $W$ boson production cross section using 20.3~fb$^{-1}$ of LHC Run 1 (8~TeV) data~\cite{Aad:2014zda}, which was recast in Ref.~\cite{Chiang:2014bia} to constrain $H_5^{\pm\pm}$ production in the GM model.\footnote{We thank Cheng-Wei Chiang for providing the numerical version of the exclusion contour of Ref.~\cite{Chiang:2014bia}.}  The recast puts an upper bound on $v_{\chi}$ (equivalently $s_H$) as a function of $m_5$ assuming BR($H_5^{++} \to W^+W^+) = 1$.  We account for the possibility that BR($H_5^{++} \to W^+W^+) < 1$ in the same way as Eq.~(\ref{eq:BRlt1}).  This constraint is accessed by calling the subroutine {\tt CALCWWJJ}, which sets the flag {\tt WWJJOK = 1} (allowed) or {\tt 0} (excluded).

\subsubsection{Drell-Yan $H_5^{\pm\pm}$ with $H_5^{\pm\pm} \to W^{\pm}W^{\pm} \to$ like-sign dileptons}

For low masses $m_5 \leq 100$~GeV, Drell-Yan production of $H_5^{++}H_5^{--}$ and $H_5^{\pm\pm} H_5^{\mp}$ with $H_5^{\pm\pm} \to W^{\pm} W^{\pm}$ is constrained by an ATLAS search for anomalous like-sign dimuon production using 20.3~fb$^{-1}$ of LHC Run 1 (8~TeV) data~\cite{ATLAS:2014kca}, which was recast in Ref.~\cite{Kanemura:2014ipa} to constrain the Higgs Triplet Model assuming degenerate $H^{++}$ and $H^+$.  The latter was reinterpreted in Ref.~\cite{Logan:2015xpa} in the GM model, assuming BR($H_5^{++} \to W^+W^+) = 1$; in this case, the measurement excludes $m_5$ values below about 76~GeV independent of $s_H$.  We take into account the possibility that BR($H_5^{++} \to W^+W^+) < 1$ by applying the upper limit on the fiducial cross section from Ref.~\cite{ATLAS:2014kca} to the quantity
\begin{eqnarray}
	\sigma_{\rm fiducial} &=& 0.95 \times \left[ \sigma_{H_5^{++}H_5^{--}} 
	\left( 2 \, {\rm BR}(H_5^{++} \to \mu^+ \mu^+) \epsilon_{H_5^{++}H_5^{--}} 
	- {\rm BR}(H_5^{++} \to \mu^+ \mu^+)^2 \epsilon_{H_5^{++}H_5^{--}}^2 \right) \right. \nonumber \\
	&& \left. + \sigma_{H_5^{++} H_5^-} {\rm BR}(H_5^{++} \to \mu^+ \mu^+) \epsilon_{H_5^{++}H_5^-}
	+ \sigma_{H_5^{--} H_5^+} {\rm BR}(H_5^{++} \to \mu^+ \mu^+) \epsilon_{H_5^{--}H_5^+}
	\right],
\end{eqnarray}
where ${\rm BR}(H_5^{++} \to \mu^+ \mu^+) = {\rm BR}(H_5^{++} \to W^+W^+) \times {\rm BR}(W^+W^+ \to \mu^+\mu^+ + {\rm MET})$.  We take the values of ${\rm BR}(W^+W^+ \to \mu^+\mu^+ + {\rm MET})$, the cross sections for the Higgs Triplet Model, and the efficiencies $\epsilon_{ij}$ from Ref.~\cite{Kanemura:2014ipa}.  The cross sections $\sigma_{ij}$ for the GM model are related to those in the Higgs Triplet Model (HTM) by~\cite{Logan:2015xpa}
\begin{equation}
	\sigma_{H_5^{++}H_5^{--}} = \sigma_{H^{++}H^{--}}^{\rm HTM}, \qquad
	\sigma_{H_5^{++}H_5^-} = \frac{1}{2} \sigma_{H^{++}H^-}^{\rm HTM}, \qquad
	\sigma_{H_5^{--}H_5^+} = \frac{1}{2} \sigma_{H^{--}H^+}^{\rm HTM}.
\end{equation} 
The values of ${\rm BR}(W^+W^+ \to \mu^+\mu^+ + {\rm MET})$ do not follow straightforwardly from the individual $W$ decay branching ratios because of quantum mechanical interference when $H_5^{++}$ is lighter than the $WW$ threshold.
The factor of 0.95 conservatively takes into account the $\pm 5\%$ theory uncertainty on the signal cross sections, computed at next-to-leading order (NLO) in QCD.
This constraint is accessed by calling the subroutine {\tt CALCLSDM}, which sets the flag {\tt LSDMOK = 1} (allowed) or {\tt 0} (excluded).

For $m_5 \geq 200$~GeV, the most sensitive search for Drell-Yan production of $H_5^{++}H_5^{--}$ with $H_5^{\pm\pm} \to W^{\pm}$ is from an ATLAS analysis with 139~fb$^{-1}$ of LHC Run 2 (13~TeV) data~\cite{ATLAS:2021jol}.  The analysis was performed in the Higgs Triplet Model, for which the production cross section of $H^{++}H^{--}$ is the same as in the GM model.  The analysis combined channels with 2 same-sign leptons, 3 leptons, and 4 leptons; the first requires that at least one of $H_5^{\pm\pm}$ decay to $W^{\pm}W^{\pm}$, while the latter two require that both do.  We thus conservatively apply the upper bound from Ref.~\cite{ATLAS:2021jol} to the quantitiy
\begin{equation}
	\sigma_{\rm eff} = 0.95 \times \sigma^{\rm NLO}_{H^{++}H^{--}} 
		\left[ {\rm BR}(H_5^{++} \to W^+W^+) \right]^2,
\end{equation}
where $\sigma^{\rm NLO}_{H^{++}H^{--}}$ is the Drell-Yan production cross section computed to NLO in QCD (provided in the HEPData files for Ref.~\cite{ATLAS:2021jol}), the factor of 0.95 conservatively accounts for the $\pm 5\%$ theoretical uncertainty on this cross section, and the two factors of ${\rm BR}(H_5^{++} \to W^+W^+)$ reflect the fact that we require that both of the doubly-charged scalars decay to $WW$.  This constraint is accessed by calling the subroutine {\tt CALCDYHPP}, which sets the flag {\tt DYHPPOK = 1} (allowed) or {\tt 0} (excluded).

\subsubsection{Drell-Yan $H_5^0 H_5^{\pm}$ with $H_5^0 \to \gamma\gamma$}

Drell-Yan production of $H_5^0 H_5^{\pm}$ with $H_5^0 \to \gamma\gamma$ is constrained by an ATLAS search for diphoton resonances in the mass range 65--600~GeV using 20.3~fb$^{-1}$ of LHC Run 1 (8~TeV) data~\cite{Aad:2014ioa} and for masses above 200~GeV using 36.7~fb$^{-1}$ of LHC Run 2 (13~TeV) data~\cite{Aaboud:2017yyg}.  We recast these searches to constrain Drell-Yan production of $H_5^0$ (in association with the mass-degenerate $H_5^{\pm}$ state) with decays to $\gamma\gamma$, which become important when $m_5$ is sufficiently below the $WW$ threshold.  We computed the cross sections for $pp \to H_5^0 H_5^+$ and $pp \to H_5^0 H_5^-$ at next-to-leading order in QCD at 8~and 13~TeV and the corresponding efficiencies $\epsilon_{ij}$ of the cuts of Ref.~\cite{Aad:2014ioa} using MadGraph5~\cite{Alwall:2014hca}.  The experimental upper limits on the fiducial cross section from Refs.~\cite{Aad:2014ioa,Aaboud:2017yyg} are then applied to the quantity
\begin{equation}
	\sigma_{\rm fiducial} = 0.95 \times \left[ \sigma_{H_5^0 H_5^+} \epsilon_{H_5^0 H_5^+} 
		+ \sigma_{H_5^0 H_5^-} \epsilon_{H_5^0 H_5^-} \right] 
		\times {\rm BR}(H_5^0 \to \gamma\gamma),
\end{equation}
where the factor of 0.95 conservatively takes into account the combined scale and PDF uncertainty on the NLO $H_5^0 H_5^{\pm}$ cross sections where this constraint is relevant.\footnote{We computed the NLO scale and PDF uncertainties using MadGraph5 and found them, combined in quadrature, to be less than $\pm 5\%$ for $m_5$ below 200~GeV at $\sqrt{s} = 8$~TeV and for $m_5$ below 300~GeV at $\sqrt{s} = 13$~TeV.}
These constraints are accessed by calling the subroutines {\tt CALCATLAS8TEVGAGA} and {\tt CALCATLAS13TEVDYGAGA}, which set the flag {\tt ATLAS8TEVGAGAOK = 1} (allowed) or {\tt 0} (excluded) and similarly for {\tt ATLAS13TEVDYGAGA}.


%%%%%%%%%%%%%%%%%%%%%%%%%%%%%%%%%%%%%%%%%%%%%%%%
\section{Using the GMCALC program}
\label{sec:using}

The GMCALC code package is available for download as a .tar.gz file from the web page
\begin{quote}
	http://people.physics.carleton.ca/$\sim$logan/gmcalc/
\end{quote}
The package includes this manual.  Feature requests and bug reports should be sent to Heather Logan at logan@physics.carleton.ca .

If the decays of $H_5^0 \to Z\gamma$, $H_5^{\pm} \to W^{\pm} \gamma$, and $H_3^{\pm} \to W^{\pm} \gamma$ are to be computed, the user must also install the LoopTools package~\cite{Hahn:1998yk} (tested with LoopTools 2.15 and 2.16), which is available from
\begin{quote}
	http://www.feynarts.de/looptools/
\end{quote}
GMCALC can be run without LoopTools, in which case the partial widths of $H_5^0 \to Z\gamma$, $H_5^{\pm} \to W^{\pm} \gamma$, and $H_3^{\pm} \to W^{\pm} \gamma$ are set to zero.

If the constraints from HiggsBounds and/or HiggsSignals are to be applied, the user must also install the HiggsBounds and HiggsSignals packages~\cite{Bechtle:2013wla} (tested with HiggsBounds 5.3.2 and HiggsSignals 2.2.3; HiggsBounds 5.9.0 and HiggsSignals 2.6.0 are NOT yet supported), which are available from 
\begin{quote}
	https://higgsbounds.hepforge.org/
\end{quote}
Using GMCALC with HiggsBounds/HiggsSignals requires compilation with gfortran version 5 or later.

The makefile for GMCALC specifies the path to the LoopTools installation, e.g.,
\begin{quote}
	LT = \$(HOME)/Documents/work/looptools/LoopTools/x86\_64-Darwin
\end{quote}
and the paths to the HiggsBounds~5 and HiggsSignals~2 installations, e.g.,
\begin{quote}
	HB5 = \$(HOME)/Documents/work/HiggsBounds-5.3.0beta
\end{quote}
These should be updated to reflect the user's system.  


\subsection{Sample main programs provided with the code}

Three sample main programs are provided with the code.  These can be used as-is, or as templates for the user to write their own programs.  The command
\begin{quotation}
	{\tt \$ make sample}
\end{quotation}
compiles the program sample.f into an executable sample.x using gfortran.  The executable is run using
\begin{quotation}
	{\tt \$ ./sample.x}
\end{quotation}
The sample programs are as follows:
\begin{itemize}
\item {\tt gmpoint.f} performs the full set of available calculations for a single parameter point and outputs the spectrum, couplings, and decay tables to the terminal.
\item {\tt gmscan.f} performs a scan over the allowed parameter ranges using the approach described in Sec.~\ref{sec:scans}.  For each scan point allowed by theoretical and indirect experimental constraints, it writes a selection of observables to a file scan\_output.data.
\item {\tt gmmg5.f} generates the files param\_card-LO.dat and param\_card-NLO.dat for use with the leading-order (LO) and next-to-leading order (NLO) Universal FeynRules Object (UFO) model files, respectively.  The NLO UFO model file can be used with the MadGraph5\_aMC@NLO framework to automatically generate Monte Carlo samples at NLO accuracy in QCD.  It also generates the file param\_card-EFTLO.dat for use with the LO UFO model file including effective vertices for the loop-induced couplings $hgg$, $h\gamma\gamma$, $hZ\gamma$, $Hgg$, $H\gamma\gamma$, $HZ\gamma$, $H_3^0gg$, $H_3^0\gamma\gamma$, $H_3^0Z\gamma$, $H_3^{\pm}W^{\mp}\gamma$, $H_5^0\gamma\gamma$, $H_5^0 Z\gamma$, and $H_5^{\pm}W^{\mp} \gamma$.
The corresponding FeynRules model files are available at \url{http://feynrules.irmp.ucl.ac.be/wiki/GeorgiMachacekModel}.
\item {\tt gmhb5.f} tests a single parameter point against the direct constraints from HiggsBounds 5 and HiggsSignals 2. It employs the subroutine {\tt CALLHBHS} which is described in Sec.~\ref{sec:direct}. The program outputs the results from HiggsBounds and HiggsSignals detailing the constraints on the the parameter point.
\end{itemize}

To run {\tt gmpoint.f}, {\tt gmscan.f}, {\tt gmmg5.f}, or {\tt gmhb5.f} without installing LoopTools, use {\tt make gmpoint-nolt}, {\tt make gmscan-nolt}, {\tt make gmmg5-nolt}, or {\tt make gmhb5-nolt}, respectively.  The executables will be {\tt gmpoint-nolt.x}, {\tt gmscan-nolt.x}, {\tt gmmg5-nolt.x}, or {\tt gmhb5-nolt.x} respectively, and in them the partial widths and/or effective couplings of $H_5^0 \to Z\gamma$, $H_5^{\pm} \to W^{\pm} \gamma$, and $H_3^{\pm} \to W^{\pm} \gamma$ will be set to zero.

\subsection{Setting the model parameters}

There are currently six choices of input parameters implemented in GMCALC:
\begin{itemize}

\item {\tt INPUTSET} $= 1$ uses the primary inputs $\mu_3^2$, $\lambda_1$, $\lambda_2$, $\lambda_3$, $\lambda_4$, $\lambda_5$, $M_1$, and $M_2$.  The parameter $\mu_2^2$ is set using the constraint on $v_{\phi}^2 + 8 v_{\chi}^2$ in terms of $G_F$.

\item {\tt INPUTSET} $= 2$ uses the primary inputs $\mu_3^2$, $m_h$, $\lambda_2$, $\lambda_3$, $\lambda_4$, $\lambda_5$, $M_1$, and $M_2$.  The parameter $\mu_2^2$ is again set using the constraint on $v_{\phi}^2 + 8 v_{\chi}^2$ in terms of $G_F$.

\item {\tt INPUTSET} $= 3$ uses the primary inputs $m_h$, $m_H$, $m_3$, $m_5$, $\sin\theta_H$, $\sin\alpha$, $M_1$, and $M_2$.  $G_F$ is also used to set $\mu_2^2$.

\item {\tt INPUTSET} $= 4$ uses the primary inputs $m_h$, $m_5$, $\sin\theta_H$, $\lambda_2$, $\lambda_3$, $\lambda_4$, $M_1$, and $M_2$.  $G_F$ is also used to set $\mu_2^2$.

\item {\tt INPUTSET} $= 5$ uses the primary inputs $m_h$, $m_H$, $\sin\theta_H$, $\sin\alpha$, $\lambda_2$, $\lambda_3$, $\lambda_4$, and $\lambda_5$.  $G_F$ is also used to set $\mu_2^2$.

\item {\tt INPUTSET} $= 6$ uses the primary inputs $m_h$, $m_5$, $\sin\theta_H$, $\lambda_2$, $\lambda_3$, $\lambda_4$, $\lambda_5$, and $M_2$.  $G_F$ is also used to set $\mu_2^2$.
\end{itemize}
These inputs can be hand-coded in the sample programs (indicated by {\tt INPUTMODE} $= 0$).  Alternatively, the program can be run in interactive mode ({\tt INPUTMODE} $= 1$) in which case the user will be prompted to enter the inputs at the terminal.  In either case, the subroutine {\tt LOAD\_INPUTS} processes the inputs and computes the remaining potential parameters.  {\tt LOAD\_INPUTS} sets a flag {\tt INPUTOK} $= 1$ if the specified inputs yield an acceptable scalar potential.

\subsection{Checking consistency and computing the spectrum}

Before computing the physical spectrum, the scalar potential should be checked for consistency with theoretical constraints.  This is accomplished by the subroutine {\tt THYCHECK}, which returns three flags: {\tt UNIOK} $= 1$ indicates that the perturbative unitarity constraints on $\lambda_{1-5}$ are satisfied; {\tt BFBOK} $= 1$ indicates that the scalar potential is bounded from below; and {\tt MINOK} $= 1$ indicates that the desired electroweak-breaking vacuum is the global minimum of the potential.

The physical masses, vevs, and custodial-singlet mixing angle $\alpha$ can then be computed by the subroutine {\tt CALCPHYS}.  Results are passed via the common block
\begin{quote}
	{\tt COMMON/PHYSPARAMS/MHL,MHH,MH3,MH5,ALPHA,VPHI,VCHI}.
\end{quote}
They can be accessed directly by adding this common block declaration in one of the sample programs; alternatively, they can be output to the terminal by the subroutine {\tt PRINT\_RESULTS} (see Sec.~\ref{sec:output}).

With the physical spectrum computed, the indirect constraints can be checked by calling the subroutine {\tt CALCINDIR}.  This returns a series of flags which, if set to $1$, indicate that the model point satisfies the corresponding indirect constraint.  The flags are: {\tt BSMMOK} ($B_s^0 \to \mu^+\mu^-$), {\tt SPAROK} (oblique $S$ parameter), {\tt BSGAMLOOSEOK} (``loose'' constraint on $b \to s \gamma$), and {\tt BSGAMTIGHTOK} (``tight'' constraint on $b \to s \gamma$).  These can be accessed directly by including the common block
\begin{quote}
      {\tt COMMON/INDIR/RBSMM,SPARAM,BSMMOK,SPAROK,BSGAMLOOSEOK,BSGAMTIGHTOK}.
\end{quote}
They are also output to the terminal by the subroutine {\tt PRINT\_RESULTS} (see Sec.~\ref{sec:output}).  The double precision variables {\tt RBSMM} and {\tt SPARAM} in this common block contain the ratio of $\overline {\rm BR}(B_s^0 \to \mu^+\mu^-)$ to its SM value and the value of the $S$ parameter for this model point, respectively.

\subsection{Computing couplings and decays}

Once {\tt CALCPHYS} has been called, we are ready to compute Higgs couplings and/or decay branching ratios.  There are three subroutines that can be called independently of each other:
\begin{itemize}
\item {\tt HLCOUPS} computes the kappa factors $\kappa_i^h$ (i.e., the couplings normalized to their SM values) of $h$.  These are output to the terminal in a tidy form by {\tt PRINT\_HCOUPS}, but can also be accessed through the common block
\begin{quote}
	{\tt COMMON/KAPPASL/KVL,KFL,KGAML,KZGAML,DKGAML,DKZGAML}.
\end{quote}

\item {\tt HHCOUPS} does the same but for $H$.  These are output to the terminal by {\tt PRINT\_HCOUPS}, but can also be accessed through the common block 
\begin{quote}
	{\tt COMMON/KAPPASH/KVH,KFH,KGAMH,KZGAMH,DKGAMH,DKZGAMH}.
\end{quote}

\item {\tt CALCDECAYS} performs the full set of partial width calculations (see Sec.~\ref{sec:decays}) for all the scalar particles in the model, as well as for the top quark, which can decay to $H_3^+ b$ if kinematically allowed.  The resulting branching ratios and total widths are output to the terminal in a tidy form by {\tt PRINT\_DECAYS}, but can also be accessed through the series of common blocks for each particle as follows:
\begin{quote}
	$h$: {\tt COMMON/HLBRS/HLBRB, HLBRTA, HLBRMU, HLBRS, HLBRC, HLBRT,
      HLBRG, HLBRGA, HLBRZGA, HLBRW, HLBRZ, 
          HLBRWH3P, HLBRZH3N,
          HLBRH3N, HLBRH3P, HLBRH5N, HLBRH5P, HLBRH5PP, HLWDTH} \\
     $H$: {\tt COMMON/HHBRS/HHBRB, HHBRTA, HHBRMU, HHBRS, HHBRC, HHBRT,
          HHBRG, HHBRGA, HHBRZGA, HHBRW, HHBRZ, 
          HHBRWH3P, HHBRZH3N,
          HHBRHL, HHBRH3N, HHBRH3P, HHBRH5N, HHBRH5P, HHBRH5PP, 
          HHWDTH} \\
          $H_3^0$: {\tt COMMON/H3NBRS/H3NBRB, H3NBRTA, H3NBRMU, H3NBRS, H3NBRC, H3NBRT,
         H3NBRZHL, H3NBRZHH, H3NBRZH5N, H3NBRWH5P,
          H3NBRG, H3NBRGA, H3NBRZGA,
          H3NWDTH} \\
          $H_3^+$: {\tt COMMON/H3PBRS/H3PBRBC, H3PBRTA, H3PBRMU, H3PBRSU,
         H3PBRCS, H3PBRTB, H3PBRBU,
          H3PBRWHL, H3PBRWHH, H3PBRZH5P, H3PBRWH5N, H3PBRWH5PP, H3PBRWGA,
          H3PWDTH} \\
          $H_5^0$: {\tt COMMON/H5NBRS/H5NBRGA, H5NBRZGA, H5NBRW, H5NBRZ,
          H5NBRZH3N, H5NBRWH3P, 
          H5NBRH3N, H5NBRH3P,
          H5NWDTH} \\
	$H_5^+$: {\tt COMMON/H5PBRS/H5PBRWZ, H5PBRZH3P, H5PBRWH3N, H5PBRH3PN, H5PBRWGA, H5PWDTH} \\
	$H_5^{++}$: {\tt COMMON/H5PPBRS/H5PPBRWW, H5PPBRWH3, H5PPBRH3P, H5PPWDTH} \\
	$t$: {\tt COMMON/TOPBRS/TOPBRW, TOPBRH3P, TOPWDTH}
\end{quote}
\end{itemize}

\subsection{Outputs}
\label{sec:output}

There are three subroutines dedicated to printing results to the terminal:
\begin{itemize}
\item {\tt PRINT\_RESULTS} prints the Lagrangian parameters, the flags indicating theoretical consistency and consistency with indirect experimental constraints, and the physical masses, vevs, and custodial-singlet mixing angle.  These must have been previously computed by calls to {\tt LOAD\_INPUTS}, {\tt THYCHECK}, {\tt CALCPHYS} and {\tt CALCINDIR} (in that order).
\item {\tt PRINT\_HCOUPS} prints the kappa factors for $h$ and $H$.  These must have been previously computed by calls to the subroutines {\tt HLCOUPS} and {\tt HHCOUPS}.  
\item {\tt PRINT\_DECAYS} prints out the decay branching ratios and total widths of all the scalars in the model, as well as those of the top quark.  These must have been previously computed by a call to the subroutine {\tt CALCDECAYS}.
\end{itemize}


\subsection{Parameter scans}
\label{sec:scans}

To perform scans over the model parameters in an efficient way, the following strategy can be adopted.  Setting $m_h$ equal to the observed Higgs boson mass $\sim 125$~GeV and setting $\mu_2^2$ using $G_F$, the seven free parameters are ({\tt INPUTSET = 2})
\begin{equation}
	\mu_3^2, \lambda_2, \lambda_3, \lambda_4, \lambda_5, M_1, \ {\rm and} \ M_2.
\end{equation}

The parameters $\lambda_3$ and $\lambda_4$ are mainly constrained by the unitarity and bounded-from-below conditions.  The allowed range of $\lambda_3$ is
\begin{equation}
	- \frac{1}{2} \pi < \lambda_3 < \frac{3}{5} \pi.
\end{equation}
The allowed range of $\lambda_4$ is then
\begin{eqnarray}
	{\rm For} \ \lambda_3 < 0: && 
		-\lambda_3 < \lambda_4 < \left( -\frac{7}{11}\lambda_3 + \frac{2}{11} \pi \right), 
		\nonumber \\ 
	{\rm For} \ \lambda_3 \geq 0: && 
		-\frac{1}{3} \lambda_3 < \lambda_4 < \left( -\frac{7}{11}\lambda_3 + \frac{2}{11} \pi \right).
\end{eqnarray}

The parameter $\lambda_2$ is constrained by the first of the unitarity constraints in Eq.~(\ref{eq:uni}).  Since we don't know $\lambda_1$ until the rest of the parameters are set, we allow it to vary to obtain the least stringent constraint (which occurs when $\lambda_1 = 0$),
\begin{equation}
	|\lambda_2| < \frac{1}{3} \sqrt{4 \pi^2 - 2 \pi (7 \lambda_3 + 11 \lambda_4)}.
\end{equation}
Note that $0 < (7 \lambda_3 + 11 \lambda_4) < 2 \pi$.  Implementing a lower bound on the scan range for $\lambda_2$ from the bounded-from-below constraint does not dramatically improve the code's efficiency.

The last of the unitarity constraints in Eq.~(\ref{eq:uni}) then constrains 
\begin{equation}
	(-2 \pi + \lambda_2) < \lambda_5 < (2 \pi + \lambda_2).
\end{equation}

The dimensionful parameters $\mu_3^2$, $M_1$, and $M_2$ are constrained by the requirement that there be an acceptable electroweak symmetry breaking vacuum.  We find that the following ranges capture all allowed parameter points:
\begin{eqnarray}
	\mu_3^2 &>& -(200~{\rm GeV})^2, \nonumber \\
	M_1 &<& {\rm max}\left(3500~{\rm GeV}, 3.5 \sqrt{|\mu_3^2|}\right), \nonumber \\
	|M_2| &<& {\rm max}\left(250~{\rm GeV}, 1.3 \sqrt{|\mu_3^2|}\right).
\end{eqnarray}
Note that $M_1$ can be chosen positive with no loss of generality, so that $0 \leq M_1$.  $M_2$ takes either sign.  There is no upper bound on $\mu_3^2$; the limit $\mu_3^2 \gg v^2$ is the decoupling limit, in which the masses-squared of the predominantly-triplet states approach $\mu_3^2$.

\subsection{Standard Model inputs}

The Standard Model input parameters are initialized by the subroutine {\tt INITIALIZE\_SM}, which must be called before anything else.  The parameter values are hard-coded in /src/gminit.f.

The primary electroweak inputs are~\cite{LHCHXSWG-INT-2015-006}
\begin{equation}
	G_F = 1.1663787 \times 10^{-5}~{\rm GeV}^{-2}, \qquad
	M_Z = 91.1876~{\rm GeV}, \qquad
	M_W = 80.385~{\rm GeV}.
\end{equation}
The SM Higgs vev is computed as $v = (\sqrt{2} G_F)^{-1/2}$.


%%%%%%%%%%%%%%%%%%%%%%%%%%%%%%%%%%%%%%%%%%%%%%%%
\section*{Acknowledgments}
We thank Tim Stefaniak for help with HiggsBounds/HiggsSignals.
C.D.\ was a Durham International Junior Research Fellow and has been supported in part by the Research Executive Agency of the European Union under Grant No.\ PITN-GA-2012-315877 (MC-Net).
K.H., A.I., B.K., K.K., H.E.L., A.D.P., and Y.W.\ were supported by the Natural Sciences and Engineering Research 
Council of Canada.  K.H.\ and B.K.\ were also supported by the Government of Ontario through an Ontario Graduate Scholarship.  H.E.L.\ acknowledges additional support from the grant H2020-MSCA-RISE-2014 No.\ 645722 (NonMinimalHiggs).

%%%%%%%%%%%%%%%%%%%%%%%%%%%%%%%%%%%%%%%%%%%%%%%%
%\appendix
%\section{Sample output}

%\section{LesHouches output}


%%%%%%%%%%%%%%%%%%%%%%%%%%%%%%%%%%%%%%%%%%%%%%%%
\begin{thebibliography}{99}

\bibitem{Georgi:1985nv} 
  H.~Georgi and M.~Machacek,
  ``Doubly Charged Higgs Bosons,''
  Nucl.\ Phys.\ B {\bf 262}, 463 (1985).
  %%CITATION = NUPHA,B262,463;%%
  
\bibitem{Chanowitz:1985ug} 
  M.~S.~Chanowitz and M.~Golden,
  ``Higgs Boson Triplets With M($W$) = M($Z$) $\cos \theta_W$,''
  Phys.\ Lett.\ B {\bf 165}, 105 (1985).
  %%CITATION = PHLTA,B165,105;%%

\bibitem{Bechtle:2013wla} 
%\bibitem{Bechtle:2008jh} 
  P.~Bechtle, O.~Brein, S.~Heinemeyer, G.~Weiglein and K.~E.~Williams,
  ``HiggsBounds: Confronting Arbitrary Higgs Sectors with Exclusion Bounds from LEP and the Tevatron,''
  Comput.\ Phys.\ Commun.\  {\bf 181}, 138 (2010)
 % doi:10.1016/j.cpc.2009.09.003
  [arXiv:0811.4169 [hep-ph]];
  %%CITATION = doi:10.1016/j.cpc.2009.09.003;%%
%\bibitem{Bechtle:2011sb} 
 % P.~Bechtle, O.~Brein, S.~Heinemeyer, G.~Weiglein and K.~E.~Williams,
  ``HiggsBounds 2.0.0: Confronting Neutral and Charged Higgs Sector Predictions with Exclusion Bounds from LEP and the Tevatron,''
  Comput.\ Phys.\ Commun.\  {\bf 182}, 2605 (2011)
 % doi:10.1016/j.cpc.2011.07.015
  [arXiv:1102.1898 [hep-ph]];
  %%CITATION = doi:10.1016/j.cpc.2011.07.015;%%
%\bibitem{Bechtle:2013gu} 
  P.~Bechtle, O.~Brein, S.~Heinemeyer, O.~St{\aa}l, T.~Stefaniak, G.~Weiglein and K.~Williams,
  ``Recent Developments in HiggsBounds and a Preview of HiggsSignals,''
  PoS CHARGED {\bf 2012}, 024 (2012)
  %doi:10.22323/1.156.0024
  [arXiv:1301.2345 [hep-ph]];
  %%CITATION = doi:10.22323/1.156.0024;%%
%\bibitem{Bechtle:2013wla} 
 % P.~Bechtle, O.~Brein, S.~Heinemeyer, O.~Stal, T.~Stefaniak, G.~Weiglein and K.~E.~Williams,
  ``$\mathsf{HiggsBounds}-4$: Improved Tests of Extended Higgs Sectors against Exclusion Bounds from LEP, the Tevatron and the LHC,''
  Eur.\ Phys.\ J.\ C {\bf 74}, no. 3, 2693 (2014)
 % doi:10.1140/epjc/s10052-013-2693-2
  [arXiv:1311.0055 [hep-ph]];
  %%CITATION = doi:10.1140/epjc/s10052-013-2693-2;%%
%\bibitem{Bechtle:2015pma} 
  P.~Bechtle, S.~Heinemeyer, O.~St{\aa}l, T.~Stefaniak and G.~Weiglein,
  ``Applying Exclusion Likelihoods from LHC Searches to Extended Higgs Sectors,''
  Eur.\ Phys.\ J.\ C {\bf 75}, no. 9, 421 (2015)
 % doi:10.1140/epjc/s10052-015-3650-z
  [arXiv:1507.06706 [hep-ph]].
  %%CITATION = doi:10.1140/epjc/s10052-015-3650-z;%%

\bibitem{Bechtle:2013xfa} 
  P.~Bechtle, S.~Heinemeyer, O.~St{\aa}l, T.~Stefaniak and G.~Weiglein,
  ``$HiggsSignals$: Confronting arbitrary Higgs sectors with measurements at the Tevatron and the LHC,''
  Eur.\ Phys.\ J.\ C {\bf 74}, no. 2, 2711 (2014)
 % doi:10.1140/epjc/s10052-013-2711-4
  [arXiv:1305.1933 [hep-ph]].
  %%CITATION = doi:10.1140/epjc/s10052-013-2711-4;%%
  
\bibitem{HKL}
%\bibitem{Hartling:2014zca} 
  K.~Hartling, K.~Kumar and H.~E.~Logan,
  ``The decoupling limit in the Georgi-Machacek model,''
  Phys.\ Rev.\ D {\bf 90}, 015007 (2014)
  [arXiv:1404.2640 [hep-ph]].
  %%CITATION = ARXIV:1404.2640;%%

\bibitem{indirect}
%\bibitem{Hartling:2014aga} 
  K.~Hartling, K.~Kumar and H.~E.~Logan,
  ``Indirect constraints on the Georgi-Machacek model and implications for Higgs boson couplings,''
  Phys.\ Rev.\ D {\bf 91}, no. 1, 015013 (2015)
  %doi:10.1103/PhysRevD.91.015013
  [arXiv:1410.5538 [hep-ph]].
  %%CITATION = doi:10.1103/PhysRevD.91.015013;%%
  
\bibitem{loops}
%\bibitem{Degrande:2017naf} 
  C.~Degrande, K.~Hartling and H.~E.~Logan,
  ``Scalar decays to $\gamma\gamma$, $Z\gamma$, and $W\gamma$ in the Georgi-Machacek model,''
  arXiv:1708.08753 [hep-ph].
  %%CITATION = ARXIV:1708.08753;%%
  
\bibitem{Degrande:2011ua}
  C.~Degrande, C.~Duhr, B.~Fuks, D.~Grellscheid, O.~Mattelaer and T.~Reiter,
  ``UFO - The Universal FeynRules Output,''
  Comput.\ Phys.\ Commun.\  {\bf 183} (2012) 1201
  [arXiv:1108.2040 [hep-ph]].
  %%CITATION = ARXIV:1108.2040;%%
  
\bibitem{Alloul:2013bka}
  A.~Alloul, N.~D.~Christensen, C.~Degrande, C.~Duhr and B.~Fuks,
  ``FeynRules  2.0 - A complete toolbox for tree-level phenomenology,''
  Comput.\ Phys.\ Commun.\  {\bf 185} (2014) 2250
  [arXiv:1310.1921 [hep-ph]].
  %%CITATION = ARXIV:1310.1921;%%

\bibitem{Alwall:2014hca}
  J.~Alwall {\it et al.},
  ``The automated computation of tree-level and next-to-leading order differential cross sections, and their matching to parton shower simulations,''
  JHEP {\bf 1407} (2014) 079
  [arXiv:1405.0301 [hep-ph]].
  %%CITATION = ARXIV:1405.0301;%%


\bibitem{Aoki:2007ah} 
  M.~Aoki and S.~Kanemura,
  ``Unitarity bounds in the Higgs model including triplet fields with custodial symmetry,''  
  Phys.\ Rev.\ D {\bf 77}, 095009 (2008)
  [arXiv:0712.4053 [hep-ph]];   
  erratum Phys.\ Rev.\ D {\bf 89}, 059902 (2014).
  %%CITATION = ARXIV:0712.4053;%%

\bibitem{PDG2018}
M.~Tanabashi {\it et al.} (Particle Data Group),
``2018 Review of Particle Physics,''
Phys.\ Rev.\ D {\bf 98}, 030001 (2018).
  
\bibitem{Gunion:1990dt} 
  J.~F.~Gunion, R.~Vega and J.~Wudka,
  ``Naturalness problems for $\rho = 1$ and other large one loop effects for a standard model Higgs sector containing triplet fields,''
  Phys.\ Rev.\ D {\bf 43}, 2322 (1991).
  %%CITATION = PHRVA,D43,2322;%%
  
\bibitem{Beringer:1900zz} 
  J.~Beringer {\it et al.}  [Particle Data Group Collaboration],
  ``Review of Particle Physics (RPP),''
  Phys.\ Rev.\ D {\bf 86}, 010001 (2012).
  %%CITATION = PHRVA,D86,010001;%%
  
\bibitem{SuperIso}
F.~Mahmoudi,
  ``SuperIso: A Program for calculating the isospin asymmetry of $B \to K^* \gamma$ in the MSSM,''
  Comput.\ Phys.\ Commun.\  {\bf 178}, 745 (2008)
  [arXiv:0710.2067 [hep-ph]];
  %%CITATION = ARXIV:0710.2067;%%
%F.~Mahmoudi,
  ``SuperIso v2.3: A Program for calculating flavor physics observables in Supersymmetry,''
  Comput.\ Phys.\ Commun.\  {\bf 180}, 1579 (2009)
  [arXiv:0808.3144 [hep-ph]];
  %%CITATION = ARXIV:0808.3144;%%
%F.~Mahmoudi,
  ``SuperIso v3.0, flavor physics observables calculations: Extension to NMSSM,''
  Comput.\ Phys.\ Commun.\  {\bf 180}, 1718 (2009).
  %%CITATION = CPHCB,180,1718;%%

\bibitem{2HDMC}
  D.~Eriksson, J.~Rathsman and O.~Stal,
  ``2HDMC: Two-Higgs-Doublet Model Calculator Physics and Manual,''
  Comput.\ Phys.\ Commun.\  {\bf 181}, 189 (2010)
  [arXiv:0902.0851 [hep-ph]];
  %%CITATION = ARXIV:0902.0851;%%
%D.~Eriksson, J.~Rathsman and O.~Stal,
  ``2HDMC: Two-Higgs-doublet model calculator,''
  Comput.\ Phys.\ Commun.\  {\bf 181}, 833 (2010).
  %%CITATION = CPHCB,181,833;%%  
  
\bibitem{Misiak:2006zs} 
  M.~Misiak, H.~M.~Asatrian, K.~Bieri, M.~Czakon, A.~Czarnecki, T.~Ewerth, A.~Ferroglia and P.~Gambino {\it et al.},
  ``Estimate of B$(\bar B \to X(s) \gamma)$ at $O(\alpha_s^2)$,''
  Phys.\ Rev.\ Lett.\  {\bf 98}, 022002 (2007)
  [hep-ph/0609232].
  %%CITATION = HEP-PH/0609232;%%

\bibitem{Li:2014fea} 
  X.-Q.~Li, J.~Lu and A.~Pich,
  ``$B_{s,d}^0 \to \ell^+\ell^-$ Decays in the Aligned Two-Higgs-Doublet Model,''
  JHEP {\bf 1406}, 022 (2014)
  [arXiv:1404.5865 [hep-ph]].
  %%CITATION = ARXIV:1404.5865;%%
  
\bibitem{bsmmexp}
CMS and LHCb Collaborations,
``Combination of results on the rare decays $B^0_{(s)} \to \mu^+ \mu^-$ from the CMS and LHCb experiments,''
 CMS-PAS-BPH-13-007, 
 available from \verb+http://cds.cern.ch+.
 
\bibitem{Djouadi:1995gt} 
  A.~Djouadi, M.~Spira and P.~M.~Zerwas,
  ``QCD corrections to hadronic Higgs decays,''
  Z.\ Phys.\ C {\bf 70}, 427 (1996)
  [hep-ph/9511344].
  %%CITATION = HEP-PH/9511344;%%
  
\bibitem{Romao:1998sr} 
  J.~C.~Romao and S.~Andringa,
  ``Vector boson decays of the Higgs boson,''
  Eur.\ Phys.\ J.\ C {\bf 7}, 631 (1999)
  [hep-ph/9807536].
  %%CITATION = HEP-PH/9807536;%%
	
\bibitem{Contino:2014aaa} 
  R.~Contino, M.~Ghezzi, C.~Grojean, M.~M\"uhlleitner and M.~Spira,
  ``eHDECAY: an Implementation of the Higgs Effective Lagrangian into HDECAY,''
  Comput.\ Phys.\ Commun.\  {\bf 185}, 3412 (2014)
  [arXiv:1403.3381 [hep-ph]].
  %%CITATION = ARXIV:1403.3381;%%
  
  \bibitem{Lepage:1980dq} 
  G.~P.~Lepage,
  ``Vegas: An Adaptive Multidimensional Integration Program,''
  CLNS-80/447.
  %%CITATION = CLNS-80/447;%%
  
  \bibitem{Beenakker:1996ed} 
  W.~Beenakker, R.~Hopker and M.~Spira,
  ``PROSPINO: A Program for the production of supersymmetric particles in next-to-leading order QCD,''
  hep-ph/9611232.
  %%CITATION = HEP-PH/9611232;%%
  
\bibitem{Djouadi:1997yw} 
  A.~Djouadi, J.~Kalinowski and M.~Spira,
  ``HDECAY: A Program for Higgs boson decays in the standard model and its supersymmetric extension,''
  Comput.\ Phys.\ Commun.\  {\bf 108}, 56 (1998)
  [hep-ph/9704448].
  %%CITATION = HEP-PH/9704448;%%
  
\bibitem{Djouadi:1995gv} 
  A.~Djouadi, J.~Kalinowski and P.~M.~Zerwas,
  ``Two and three-body decay modes of SUSY Higgs particles,''
  Z.\ Phys.\ C {\bf 70}, 435 (1996)
  [hep-ph/9511342].
  %%CITATION = HEP-PH/9511342;%%
  
\bibitem{Akeroyd:1998dt} 
  A.~G.~Akeroyd,
 ``Three body decays of Higgs bosons at LEP-2 and application to a hidden fermiophobic Higgs,''
  Nucl.\ Phys.\ B {\bf 544}, 557 (1999)
  [hep-ph/9806337].
  %%CITATION = HEP-PH/9806337;%%

\bibitem{HHG}
J.~F.~Gunion, H.~E.~Haber, G.~L.~Kane, and S.~Dawson, 
{\it The Higgs Hunter's Guide} (Westview, Boulder, Colorado, 2000).
%%CITATION = FRPHA,80,1;%%

\bibitem{Hahn:1998yk} 
  T.~Hahn and M.~Perez-Victoria,
  ``Automatized one loop calculations in four-dimensions and D-dimensions,''
  Comput.\ Phys.\ Commun.\  {\bf 118}, 153 (1999)
  %doi:10.1016/S0010-4655(98)00173-8
  [hep-ph/9807565].
  %%CITATION = doi:10.1016/S0010-4655(98)00173-8;%%

\bibitem{CMS:2021wlt}
A.~M.~Sirunyan \textit{et al.} [CMS],
``Search for charged Higgs bosons produced in vector boson fusion processes and decaying into vector boson pairs in proton--proton collisions at $\sqrt{s} = 13~{\rm TeV}$,''
Eur. Phys. J. C \textbf{81}, no.8, 723 (2021)
%doi:10.1140/epjc/s10052-021-09472-3
[arXiv:2104.04762 [hep-ex]].

\bibitem{Sirunyan:2017ret} 
  A.~M.~Sirunyan {\it et al.} [CMS Collaboration],
  ``Observation of electroweak production of same-sign W boson pairs in the two jet and two same-sign lepton final state in proton-proton collisions at $\sqrt{s} = $ 13 TeV,''
  Phys.\ Rev.\ Lett.\  {\bf 120}, no. 8, 081801 (2018)
  %doi:10.1103/PhysRevLett.120.081801
  [arXiv:1709.05822 [hep-ex]].
  %%CITATION = doi:10.1103/PhysRevLett.120.081801;%%

\bibitem{Aad:2014zda} 
  G.~Aad {\it et al.} [ATLAS Collaboration],
  ``Evidence for Electroweak Production of $W^{\pm}W^{\pm}jj$ in $pp$ Collisions at $\sqrt{s}=8$ TeV with the ATLAS Detector,''
  Phys.\ Rev.\ Lett.\  {\bf 113}, no. 14, 141803 (2014)
  %doi:10.1103/PhysRevLett.113.141803
  [arXiv:1405.6241 [hep-ex]].
  %%CITATION = doi:10.1103/PhysRevLett.113.141803;%%

\bibitem{Chiang:2014bia} 
  C.~W.~Chiang, S.~Kanemura and K.~Yagyu,
  ``Novel constraint on the parameter space of the Georgi-Machacek model with current LHC data,''
  Phys.\ Rev.\ D {\bf 90}, no. 11, 115025 (2014)
  %doi:10.1103/PhysRevD.90.115025
  [arXiv:1407.5053 [hep-ph]].
  %%CITATION = doi:10.1103/PhysRevD.90.115025;%%
      
\bibitem{ATLAS:2014kca} 
  G.~Aad {\it et al.} [ATLAS Collaboration],
  ``Search for anomalous production of prompt same-sign lepton pairs and pair-produced doubly charged Higgs bosons with $ \sqrt{s}=8 $ TeV $pp$ collisions using the ATLAS detector,''
  JHEP {\bf 1503}, 041 (2015)
  %doi:10.1007/JHEP03(2015)041
  [arXiv:1412.0237 [hep-ex]].
  %%CITATION = doi:10.1007/JHEP03(2015)041;%%

\bibitem{Kanemura:2014ipa} 
  S.~Kanemura, M.~Kikuchi, H.~Yokoya and K.~Yagyu,
  ``LHC Run-I constraint on the mass of doubly charged Higgs bosons in the same-sign diboson decay scenario,''
  PTEP {\bf 2015}, 051B02 (2015)
  %doi:10.1093/ptep/ptv071
  [arXiv:1412.7603 [hep-ph]].
  %%CITATION = doi:10.1093/ptep/ptv071;%%

\bibitem{Logan:2015xpa} 
  H.~E.~Logan and V.~Rentala,
  ``All the generalized Georgi-Machacek models,''
  Phys.\ Rev.\ D {\bf 92}, no. 7, 075011 (2015)
  %doi:10.1103/PhysRevD.92.075011
  [arXiv:1502.01275 [hep-ph]].
  %%CITATION = doi:10.1103/PhysRevD.92.075011;%%
  
\bibitem{ATLAS:2021jol}
G.~Aad \textit{et al.} [ATLAS],
``Search for doubly and singly charged Higgs bosons decaying into vector bosons in multi-lepton final states with the ATLAS detector using proton-proton collisions at $ \sqrt{\mathrm{s}} $ = 13 TeV,''
JHEP \textbf{06}, 146 (2021)
%doi:10.1007/JHEP06(2021)146
[arXiv:2101.11961 [hep-ex]].
%33 citations counted in INSPIRE as of 03 Jun 2022

\bibitem{Aad:2014ioa} 
  G.~Aad {\it et al.} [ATLAS Collaboration],
  ``Search for Scalar Diphoton Resonances in the Mass Range $65-600$ GeV with the ATLAS Detector in $pp$ Collision Data at $\sqrt{s}$ = 8 $TeV$,''
  Phys.\ Rev.\ Lett.\  {\bf 113}, no. 17, 171801 (2014)
  %doi:10.1103/PhysRevLett.113.171801
  [arXiv:1407.6583 [hep-ex]].
  %%CITATION = doi:10.1103/PhysRevLett.113.171801;%%
  
\bibitem{Aaboud:2017yyg} 
  M.~Aaboud {\it et al.} [ATLAS Collaboration],
  ``Search for new phenomena in high-mass diphoton final states using 37 fb$^{-1}$ of proton--proton collisions collected at $\sqrt{s}=13$ TeV with the ATLAS detector,''
  Phys.\ Lett.\ B {\bf 775}, 105 (2017)
  %doi:10.1016/j.physletb.2017.10.039
  [arXiv:1707.04147 [hep-ex]].
  %%CITATION = doi:10.1016/j.physletb.2017.10.039;%%
        
\bibitem{LHCHXSWG-INT-2015-006}
A.~Denner, S.~Dittmaier, M.~Grazzini, R.~V.~Harlander, R.~S.~Thorne, M.~Spira, and M.~Steinhauser, 
``Standard Model input parameters for Higgs physics,''
LHCHXSWG-INT-2015-006, available from
\verb+https://cds.cern.ch/record/2047636+.
  
\end{thebibliography}
\end{document}
